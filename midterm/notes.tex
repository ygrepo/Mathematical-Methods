\documentclass[12pt,twoside]{article}
\usepackage[dvipsnames]{xcolor}
\usepackage{tikz,graphicx,amsmath,amsfonts,amscd,amssymb,bm,cite,epsfig,epsf,url}
\usepackage[hang,flushmargin]{footmisc}
\usepackage[colorlinks=true,urlcolor=blue,citecolor=blue]{hyperref}
\usepackage{amsthm,multirow,wasysym,appendix}
\usepackage{array,subcaption} 
% \usepackage[small,bf]{caption}
\usepackage{bbm}
\usepackage{pgfplots}
\usetikzlibrary{spy}
\usepgfplotslibrary{external}
\usepgfplotslibrary{fillbetween}
\usetikzlibrary{arrows,automata}
\usepackage{thmtools}
\usepackage{blkarray} 
\usepackage{textcomp}
\usepackage[left=0.8in,right=1.0in,top=1.0in,bottom=1.0in]{geometry}

\usepackage{times}
\usepackage{amsfonts}
\usepackage{amsmath}
\usepackage{latexsym}
\usepackage{commath}
\usepackage{color}
\usepackage{graphics}
\usepackage{enumerate}
\usepackage{amstext}
\usepackage{blkarray}
\usepackage{url}
\usepackage{epsfig}
\usepackage{bm}
\usepackage{hyperref}
\hypersetup{
    colorlinks=true,
    linkcolor=blue,
    filecolor=magenta,      
    urlcolor=blue,
}
\usepackage{textcomp}
\usepackage[left=0.8in,right=1.0in,top=1.0in,bottom=1.0in]{geometry}
\usepackage{mathtools}
\usepackage{minted}

%% Probability operators and functions
%
% \def \P{\mathrm{P}}
\def \P{\mathrm{P}}
\def \E{\mathrm{E}}
\def \Var{\mathrm{Var}}
\let\var\Var
\def \Cov {\mathrm{Cov}} \let\cov\Cov
\def \MSE {\mathrm{MSE}} \let\mse\MSE
\def \sgn {\mathrm{sgn}}
\def \R {\mathbb{R}}
\def \C {\mathbb{C}}
\def \N {\mathbb{N}}
\def \Z {\mathbb{Z}}
\def \cV {\mathcal{V}}
\def \cS {\mathcal{S}}

\newcommand{\RR}{\ensuremath{\mathbb{R}}}

\DeclareMathOperator*{\argmin}{arg\,min}
\DeclareMathOperator*{\argmax}{arg\,max}
\newcommand{\red}[1]{\textcolor{red}{#1}}
\newcommand{\blue}[1]{\textcolor{blue}{#1}}
\newcommand{\green}[1]{\textcolor{ForestGreen}{ #1}}
\newcommand{\fuchsia}[1]{\textcolor{RoyalPurple}{ #1}}



%
%% Probability distributions
%
%\def \Bern    {\mathrm{Bern}}
%\def \Binom   {\mathrm{Binom}}
%\def \Exp     {\mathrm{Exp}}
%\def \Geom    {\mathrm{Geom}}
% \def \Norm    {\mathcal{N}}
%\def \Poisson {\mathrm{Poisson}}
%\def \Unif    {\mathrm {U}}
%
\DeclareMathOperator{\Norm}{\mathcal{N}}

\newcommand{\bdb}[1]{\textcolor{red}{#1}}

\newcommand{\ml}[1]{\mathcal{ #1 } }
\newcommand{\wh}[1]{\widehat{ #1 } }
\newcommand{\wt}[1]{\widetilde{ #1 } }
\newcommand{\conj}[1]{\overline{ #1 } }
\newcommand{\rnd}[1]{\tilde{ #1 } }
\newcommand{\rv}[1]{ \rnd{ #1}  }
\newcommand{\rx}{\rnd{ x}  }
\newcommand{\ry}{\rnd{ y}  }
\newcommand{\rz}{\rnd{ z}  }
\newcommand{\ra}{\rnd{ a}  }
\newcommand{\rb}{\rnd{ b}  }
\newcommand{\rpc}{\widetilde{ pc}  }
\newcommand{\rndvec}[1]{\vec{\rnd{#1}}}

\def \cnd {\, | \,}
\def \Id { I }
\def \J {\mathbf{1}\mathbf{1}^T}

\newcommand{\op}[1]{\operatorname{#1}}
\newcommand{\setdef}[2]{ := \keys{ #1 \; | \; #2 } }
%\newcommand{\set}[2]{ \keys{ #1 \; | \; #2 } }
\newcommand{\sign}[1]{\op{sign}\left( #1 \right) }
\newcommand{\trace}[1]{\op{tr}\left( #1 \right) }
\newcommand{\tr}[1]{\op{tr}\left( #1 \right) }
\newcommand{\inv}[1]{\left( #1 \right)^{-1} }
%\newcommand{\abs}[1]{\left| #1 \right|}
\newcommand{\sabs}[1]{| #1 |}
\newcommand{\keys}[1]{\left\{ #1 \right\}}
\newcommand{\sqbr}[1]{\left[ #1 \right]}
\newcommand{\sbrac}[1]{ ( #1 ) }
\newcommand{\brac}[1]{\left( #1 \right) }
\newcommand{\bbrac}[1]{\big( #1 \big) }
\newcommand{\Bbrac}[1]{\Big( #1 \Big)}
\newcommand{\BBbrac}[1]{\BIG( #1 \Big)}
\newcommand{\MAT}[1]{\begin{bmatrix} #1 \end{bmatrix}}
\newcommand{\sMAT}[1]{\left(\begin{smallmatrix} #1 \end{smallmatrix}\right)}
\newcommand{\sMATn}[1]{\begin{smallmatrix} #1 \end{smallmatrix}}
\newcommand{\PROD}[2]{\left \langle #1, #2\right \rangle}
\newcommand{\PRODs}[2]{\langle #1, #2 \rangle}
\newcommand{\der}[2]{\frac{\text{d}#2}{\text{d}#1}}
\newcommand{\pder}[2]{\frac{\partial#2}{\partial#1}}
\newcommand{\derTwo}[2]{\frac{\text{d}^2#2}{\text{d}#1^2}}
\newcommand{\ceil}[1]{\lceil #1 \rceil}
\newcommand{\Imag}[1]{\op{Im}\brac{ #1 }}
\newcommand{\Real}[1]{\op{Re}\brac{ #1 }}
%\newcommand{\norm}[1]{\left|\left| #1 \right|\right| }
\newcommand{\norms}[1]{ \| #1 \|  }
\newcommand{\normProd}[1]{\left|\left| #1 \right|\right| _{\PROD{\cdot}{\cdot}} }
\newcommand{\normTwo}[1]{\left|\left| #1 \right|\right| _{2} }
\newcommand{\normTwos}[1]{ \| #1  \| _{2} }
\newcommand{\normZero}[1]{\left|\left| #1 \right|\right| _{0} }
\newcommand{\normTV}[1]{\left|\left| #1 \right|\right|  _{ \op{TV}  } }% _{\op{c} \ell_1} }
\newcommand{\normOne}[1]{\left|\left| #1 \right|\right| _{1} }
\newcommand{\normOnes}[1]{\| #1 \| _{1} }
\newcommand{\normOneTwo}[1]{\left|\left| #1 \right|\right| _{1,2} }
\newcommand{\normF}[1]{\left|\left| #1 \right|\right| _{\op{F}} }
\newcommand{\normLTwo}[1]{\left|\left| #1 \right|\right| _{\ml{L}_2} }
\newcommand{\normNuc}[1]{\left|\left| #1 \right|\right| _{\ast} }
\newcommand{\normOp}[1]{\left|\left| #1 \right|\right|  }
\newcommand{\normInf}[1]{\left|\left| #1 \right|\right| _{\infty}  }
\newcommand{\proj}[1]{\mathcal{P}_{#1} \, }
\newcommand{\diff}[1]{ \, \text{d}#1 }
\newcommand{\vc}[1]{\boldsymbol{\vec{#1}}}
\newcommand{\rc}[1]{\boldsymbol{#1}}
\newcommand{\vx}{\vec{x}}
\newcommand{\vy}{\vec{y}}
\newcommand{\vz}{\vec{z}}
\newcommand{\vu}{\vec{u}}
\newcommand{\vv}{\vec{v}}
\newcommand{\vb}{\vec{\beta}}
\newcommand{\va}{\vec{\alpha}}
\newcommand{\vaa}{\vec{a}}
\newcommand{\vbb}{\vec{b}}
\newcommand{\vg}{\vec{g}}
\newcommand{\vw}{\vec{w}}
\newcommand{\vh}{\vec{h}}
\newcommand{\vbeta}{\vec{\beta}}
\newcommand{\valpha}{\vec{\alpha}}
\newcommand{\vgamma}{\vec{\gamma}}
\newcommand{\veta}{\vec{\eta}}
\newcommand{\vnu}{\vec{\nu}}
\newcommand{\rw}{\rnd{w}}
\newcommand{\rvnu}{\vc{\nu}}
\newcommand{\rvv}{\rndvec{v}}
\newcommand{\rvw}{\rndvec{w}}
\newcommand{\rvx}{\rndvec{x}}
\newcommand{\rvy}{\rndvec{y}}
\newcommand{\rvz}{\rndvec{z}}
\newcommand{\rvX}{\rndvec{X}}


\newtheorem{theorem}{Theorem}[section]
% \declaretheorem[style=plain,qed=$\square$]{theorem}
\newtheorem{corollary}[theorem]{Corollary}
\newtheorem{definition}[theorem]{Definition}
\newtheorem{lemma}[theorem]{Lemma}
\newtheorem{remark}[theorem]{Remark}
\newtheorem{algorithm}[theorem]{Algorithm}

% \theoremstyle{definition}
%\newtheorem{example}[proof]{Example}
\declaretheorem[style=definition,qed=$\triangle$,sibling=definition]{example}
\declaretheorem[style=definition,qed=$\bigcirc$,sibling=definition]{application}

%
%% Typographic tweaks and miscellaneous
%\newcommand{\sfrac}[2]{\mbox{\small$\displaystyle\frac{#1}{#2}$}}
%\newcommand{\suchthat}{\kern0.1em{:}\kern0.3em}
%\newcommand{\qqquad}{\kern3em}
%\newcommand{\cond}{\,|\,}
%\def\Matlab{\textsc{Matlab}}
%\newcommand{\displayskip}[1]{\abovedisplayskip #1\belowdisplayskip #1}
%\newcommand{\term}[1]{\emph{#1}}
%\renewcommand{\implies}{\;\Rightarrow\;}

% My macros

\def\Kset{\mathbb{K}}
\def\Nset{\mathbb{N}}
\def\Qset{\mathbb{Q}}
\def\Rset{\mathbb{R}}
\def\Sset{\mathbb{S}}
\def\Zset{\mathbb{Z}}
\def\squareforqed{\hbox{\rlap{$\sqcap$}$\sqcup$}}
\def\qed{\ifmmode\squareforqed\else{\unskip\nobreak\hfil
\penalty50\hskip1em\null\nobreak\hfil\squareforqed
\parfillskip=0pt\finalhyphendemerits=0\endgraf}\fi}

%\DeclareMathOperator*{\E}{\rm E}
%\DeclareMathOperator*{\argmax}{\rm argmax}
%\DeclareMathOperator*{\argmin}{\rm argmin}
%\DeclareMathOperator{\sgn}{sign}
\DeclareMathOperator{\supp}{supp}
\DeclareMathOperator{\last}{last}
%\DeclareMathOperator{\sign}{\sgn}
\DeclareMathOperator{\diag}{diag}
\providecommand{\abs}[1]{\lvert#1\rvert}
\providecommand{\norm}[1]{\lVert#1\rVert}
\def\vcdim{\textnormal{VCdim}}
\DeclareMathOperator*{\B}{\textbf{B}}

%\DeclarePairedDelimiter\ceil{\lceil}{\rceil}
%\DeclarePairedDelimiter\floor{\lfloor}{\rfloor}

\newcommand{\cX}{{\mathcal X}}
\newcommand{\cY}{{\mathcal Y}}
\newcommand{\cA}{{\mathcal A}}
\newcommand{\ignore}[1]{}
\newcommand{\ba}{\[\begin{aligned}}
\newcommand{\ea}{\end{aligned}\]}
\newcommand{\bi}{\begin{itemize}}
\newcommand{\ei}{\end{itemize}}
\newcommand{\be}{\begin{enumerate}}
\newcommand{\ee}{\end{enumerate}}
\newcommand{\bd}{\begin{description}}
\newcommand{\ed}{\end{description}}
\newcommand{\h}{\widehat}
\newcommand{\e}{\epsilon}
\newcommand{\mat}[1]{{\mathbf #1}}
%\newcommand{\R}{\mat{R}}
\newcommand{\0}{\mat{0}}
\newcommand{\M}{\mat{M}}

\newcommand{\D}{\mat{D}}
\renewcommand{\r}{\mat{r}}
\newcommand{\x}{\mat{x}}
\renewcommand{\u}{\mat{u}}
\renewcommand{\v}{\mat{v}}
\newcommand{\w}{\mat{w}}
\renewcommand{\H}{\text{0}}
\newcommand{\T}{\text{1}}
%\newcommand{\set}[1]{\{#1\}}
\newcommand{\xxi}{{\boldsymbol \xi}}
\newcommand{\ssigma}{{\boldsymbol \sigma}}
\newcommand{\Alpha}{{\boldsymbol \alpha}}
\newcommand{\tts}{\tt \small}
\newcommand{\hint}{\emph{hint}}
\newcommand{\matr}[1]{\bm{#1}}     % ISO complying version
\newcommand{\vect}[1]{\bm{#1}} % vectors

%\newcommand{\Var}{\mathrm{Var}}
%\newcommand{\Cov}{\mathrm{Cov}}

% New commands
\newcommand{\SP}{\mathbf{S}_{+}^n}
\newcommand{\Proj}{\mathcal{P}_{\mathcal{S}}}
%\DeclarePairedDelimiterX{\inp}[2]{\langle}{\rangle}{#1, #2}


\title{Notes Modules 1-7}

\begin{document}
%\maketitle

\section{Fourier Series}

Fourier series representation

\[
	f(x) = \frac{a_0}{2} + \sum_{r=1}^{\infty} \bigg [  a_r \cos ( \frac{2 \pi r x}{L} ) + b_r \sin ( \frac{2 \pi r x}{L} ) \bigg ]
\]
with
\ba
	a_0 &= \frac{2}{L} \int_{x_0}^{x_0+L} f(x) dx \\
	a_r &= \frac{2}{L} \int_{x_0}^{x_0+L} f(x) \cos \bigg (  \frac{2 \pi r x}{L}  \bigg) dx \\
	b_r &= \frac{2}{L} \int_{x_0}^{x_0+L} f(x) \sin \bigg (  \frac{2 \pi r x}{L}  \bigg) dx \\
\ea
For an even function, we have:
\[
	a_0 = \frac{4}{T} \int_{0}^{\frac{L}{2}} f(x) dx \;\; a_r = \frac{4}{T} \int_{0}^{\frac{L}{2}} f(x)   \cos \bigg (  \frac{2 \pi r x}{L}  \bigg)  dx \;\; b_r = 0
\]
For an odd function, we have:
\[
	a_0 = 0 \;\; b_r = \frac{4}{T} \int_{0}^{\frac{L}{2}} f(x)   \sin \bigg (  \frac{2 \pi r x}{L}  \bigg)  dx
\]

Complex Fourier series

\[
	f(x) = \sum_{r=-\infty}^{r=\infty} c_r e^{ \frac{i 2 \pi r x} {L}}
\]
where 
\ba
	c_r &= \frac{1}{L} \int_{x_0}^{x_0 + L} f(x) e^{- \frac{i 2 \pi r x} {L}} dx \\
	c_r &= \frac{1}{2} (a_r - i b_r)
\ea
Parseval Identity

\[
	\frac{1}{L}  \int_{x_0}^{x_0 + L} |f(x)|^2 dx = \sum_{r=-\infty}^{r=\infty} |c_r|^2 
\]
\section{Fourier transforms}

Fourier transform of $f(t)$:
\[	
	\tilde{f}(w) = \frac{1}{\sqrt{2 \pi}} \int_{-\infty}^{\infty} f(t) e^{-i w t} \, dt
\]
And its inverse defined by:
\[	
	f(t) = \frac{1}{\sqrt{2 \pi}} \int_{-\infty}^{\infty} \tilde{f}(w) e^{i w t} \, dw
\]

\section{The Dirac $\delta$-Function}

$\delta(t) = 0$ for $t = 0$.
Provided the range of integration includes the point $t=a$:
\[
	\int f(t) \delta(t-a) \, dt = f(a)
\] otherwise the integral equals $0$.
This leads to:

\ba
	\int_{-\infty}^{\infty} \delta(t) f(t) dt &= f(0) \\
	\int_{-a}^b \delta(t) \, dt &= 1  ~ \text{ for all } a, b > 0 \\
	\int \delta(t-a) \, dt &= 1 ~ \text{if range of integration includes a}\\	
	\delta(t) &= \delta(-t) \\
	\delta(b t) &= \frac{1}{|b|} \delta(t) \\
	t \delta(t) &= 0 \\
	\delta(h(t)) &= \sum_i \frac{\delta(t - t_i)} {|h'(t_i)|} ~ \text{where the $t_i$ are the zeros of $h(t)$}
\ea

The derivatives $\delta^n(t)$ are defined by: 
\[
	\int_{-\infty}^{\infty} f(t) \delta^n (t) \, dt = (-1)^n f^n (0)
\]


Integral representation:
\[
\delta(t - u) = \frac{1}{2 \pi} \int_{-\infty}^{\infty}  e^{i w (t - u)} \, dw
\]

Dirac function's relation to the sinc function:
\[
	\delta(t) = \lim_{\Omega \rightarrow \infty} \frac{ \sin{(\Omega t) }} { \pi t}
\]

\section{The Heaviside function}

\[
 H(t) = 
  \begin{cases} 
   1 & \text{if } t > 0 \\
   0       & \text{if } t < 0
  \end{cases}
\]

$H'(t) = \delta(t)$.


\section{Properties of Fourier transforms}

\bi
	\item $L \{ {t^n} \} = \frac{n!}{s^{n+1}}$
	\item $F \{e^{\alpha t} f(t)  \} =  \tilde{f}(w + \alpha t)$
	\item $F \{ f(t + a) \} = e^{i w a} \tilde{f}(w)$
	\item $F \{ f(a t) \} = \frac{1}{a} \tilde{f}(\frac{w}{a})$	
	\item $\ml{F} \big[ f^n(t) \big] = (i)^n w^n \tilde{f}(w)$
	\item $\ml{F} \big[ \int^t f(s) ds \big] = \frac{1}{i w} \tilde{f(w)} + 2 \pi c \delta(w)$
\ei

The Fourier sine transform: $ \tilde{f_s}(w) = \sqrt{\frac{2}{\pi}} \int_0^\infty f(t) \sin(w t) dt$

\section{Parseval's theorem}

\[
	\int_{-\infty}^{\infty} |f(x)|^2 \, dx = \int_{-\infty}^{\infty} |\tilde{f}(k)|^2 dk 
\]

\section{Laplace transform}
By definition:
\[
	\bar{f}(s) = \int_{0}^{\infty} f(t) e^{-st} \, dt
\]
And
\[
	\ml{L} [t^n f(t)] = (-1)^n \frac{d^n \bar{f}(s)} {ds^n} %~ \text{for} n =1 ,2 ,3,...
\]

Laplace transform properties:
\bi
	\item $L \{ e^{a t} f(t)  \} = \bar{f} (s-a)$ or $L^{-1} \{ \bar{f} (s-a)\} = e^{a t} f(t) $
	\item  $L \{ f(t-b) H(t-b) \} = e^{-b s} \bar{f} (s)$ or $(t-b) H(t-b) = L^{-1}\{e^{-bs} \bar{f}(s)\}$
	\item $L\{f'(t)\} = -f(0) + s \tilde{f}(s)$
	\item $L\{f''(t)\} = -s f(0) -f'(0) + s^2 \tilde{f}(s)$
	\item $L\{ \int_0^t f(x) dx \} = \frac{1}{s} \tilde{f}(s)$
\ei

\section{Convolution}

\[
	h(z) = \int_{-\infty}^{\infty} f(x) g(z-x) dx
\]

Fourier transform of a convolution: $\tilde{h}(k) = \sqrt{2\pi} \tilde{f}(k) \tilde{g}(k)$\\
Laplace transform of a convolution: $\ml{L}\big[\int_0^t f(u) g(t-u) du \big] = \bar{f}(s) \bar{g}(s)$

\section{Legendre Linear differential equations}

\[
	a_n (\alpha x + \beta)^n \frac{d^n \, y}{dx^n} + \cdots + a_1 (\alpha x + \beta) \frac{d \, y}{dx} + a_0 y = f(x)
\]
Change of variable: $\alpha x + \beta = e^t$ or $x=(e^t - \beta) / \alpha$ or $t=\ln(\alpha x + \beta)$.
Relations:

\bi
	\item $(\alpha x + \beta)  \frac{dy}{dx} = \alpha \frac{dy}{dt}$
	\item $(\alpha x + \beta)^2  \frac{d^2y}{dx^2} = \alpha^2 \frac{d}{dt} \bigg( \frac{d}{dt} - 1 \bigg) y$
\ei

\section{Variation of parameters}
Used to find a complementary solution to the homogeneous equation: $y_p(x) = k_1(x) y_1(x) + k_2(x) y_2(x) + \cdots + k_n(x) y_n(x)$
\ba
	k'_1(x) y_1(x) + k'_2(x) y_2(x) + \cdots + k'_n(x) y_n(x) &= 0 \\
	k'_1(x) y'_1(x) + k'_2(x) y'_2(x) + \cdots + k'_n(x) y'_n(x) &= 0 \\
	\vdots \\
	k'_1(x) y^{n-1}_1(x) + k'_2(x) y^{n-1}_2(x) + \cdots + k'_n(x) y^{n-1}_n(x) &= \frac{f(x)}{a_n(x)}
\ea

\section{Green function}
Looking for a solution of a differential equation with some boundary conditions on some interval $(a,b)$, e.g. $y(a) = y(b) = 0$ of the form: $y(x) = \int_a^b G(x,z) f(z) dz$.
\ba
	\ml{L} y(x) &= \ml{L} \int_a^b G(x,z) f(z) dz = \int_a^b \ml{L}G(x,z) f(z) dz = f(x) = \int_a^b \delta(x-z) f(z) dz \\
	\ml{L}G(x,z) &=  \delta(x-z) \\
\ea

Green's function properties for an nth-order linear ODE
\bi
\item Obeys the boundary conditions on $y(x)$: $G(a,z) = G(b, z) = 0$
\item The derivatives of $G(x,z)$ w.r.t. $x$ up to order $n-2$ are continuous a $x = z$
\item But the $n-1$th-order derivatives has a discontinuity of $\frac{1}{a_n}$ at this point.
\ei
  
\section{Wronskian}
Consider the linear ODE in standard form:  $y'' + p(x) y' + q(x) y = 0$.
Two solutions $y_1(x)$ and $y_2(x)$ for the solution $y(x) = c_1 y_1(x) + c_2 y_2(x)$ of the differential homogeneous equation are independant if
the Wronskian is non zero:
\ba
	W &= \begin{vmatrix} y_1(x) & y_2(x) \\  y'_1(x) & y'_2(x)  \end{vmatrix} \neq 0 \\
	W &= C e^{- \int^x p(u) du} \\
\ea

\section{Singular point}

$p(z)$ and $q(z)$ diverge at $z=z_0$, $z_0$ is a regular singular point if $(z-z_0) p(z)$ and $(z-z_0)^2 q(z)$ are both analytic.
For $\pm \infty$, make the change of variable $z = \frac{1}{w}$ and investigate if $p(z)$ and $q(z)$ are defined at $w=0$.

\section{Series solution of ODE at ordinary point}
The second order ODE solution includes two arbitrary constants and the series solution method generates two linearly independent solutions.
Solution can be represented as: $y(z) = \sum_{n=0}^\infty a_n z^n$
\ba
	y'(z) &=  \sum_{n=0}^\infty n a_n z^{n-1} =  \sum_{n=1}^\infty n a_n z^{n-1}  = \sum_{n=0}^\infty (n+1) a_{n+1} z^n\\
	y''(z) &=  \sum_{n=0}^\infty n (n-1) a_n z^{n-2} =  \sum_{n=2}^\infty n (n-1) a_n z^{n-2}  = \sum_{n=0}^\infty (n+2) (n+1) a_{n+2} z^n \\
\ea

\section{Series solution of ODE at regular singular point}

We look for a solution of the form: $y(z) = z^\sigma \sum_{n=0}^\infty a_n z^n$ where the $a_n$ and $\sigma$ are constants.

\ba
	y(z) &= \sum_{n=0}^\infty a_n z^{\sigma + n} \\
	y'(z) &= \sum_{n=0}^\infty (\sigma + n) a_n z^{\sigma + n - 1} \\
	y(z) &= \sum_{n=0}^\infty (\sigma + n-1) (\sigma + n) a_n z^{\sigma + n - 2} \\
\ea
Find the roots of the indicial equation $\sigma_1 \ge \sigma_2$:
\bi
	\item the indicial equation has two different roots where their difference is not an integer, find recurrence relation
	\bi
		\item calculation of $y(z, \sigma_2)$
		\item second solution of the form: $y_2(z) = z^{\sigma_2} \sum_{n=0}^{\infty} a_n z^n$
	\ei
	\item double root: derivative or Wronskian method: 
	\bi
		\item calculation of $\bigg [ \frac{\partial y(z,\sigma)}{ \partial \sigma} \bigg ]_{\sigma=\sigma_1} \Rightarrow y_2(z) = y_1(z) \ln z +  z^{\sigma_1} \sum_{n=0}^{\infty} b_n z^n$
		\item or Wronskian method and $y_2(z) = y_1(z) \int^z \frac{g(u)} {y_1^2(u)} du$ where $g(u) = \exp \big \{  - \int^u p(v) dv \big \}$
	\ei
	\item two roots where their difference is an integer: 
	\bi
		\item $\bigg [ \frac{\partial [ (\sigma - \sigma_2) y(z,\sigma) ]}{ \partial \sigma} \bigg ]_{\sigma=\sigma_2} \Rightarrow y_2(z) = c y_1(z) \ln z +  z^{\sigma_2} \sum_{n=0}^{\infty} b_n z^n $
		\item or Wronskian method and $y_2(z) = y_1(z) \int^z \frac{g(u)} {y_1^2(u)} du$ where $g(u) = \exp \big \{  - \int^u p(v) dv \big \}$
	\ei
\ei

\section{Series solution of an ODE at ordinary poit with arbitrary coefficient in its no-differential term}
Polynomial solutions of ODEs
\[
	y'' -2 z y' + \lambda y = 0
\]

\section{Strum-Liouville differential equations}
The standard form for a Strum-Liouville differential equation:
\[
	p(x) \frac{d^2y}{dx^2} + r(x) \frac{dy}{dx} + q(x) y + \lambda \rho(x) y = 0 \; r(x) = \frac{dp(x)}{dx}
\]
In case $r(x) \neq \frac{dp(x)}{dx}$, compute $F(x) = e^{\int^x \frac{r(u) -p'(u)}{p(u)}} du$ and multiple the differential equation by it.

\section{Legendree Functions}

Legendre's defining differential equation:
\[
	(1-x^2) y'' - 2 x y' + l (l+1) y = 0
\]

Solution of Legendre's differential equation:
\[
	y(x) = c_1 P_l(x) + c_2 Q_l(x)
\]

Rodrigues' formula:
\[
	p_l(x) = \frac{1}{l!2^l} \frac{d^l}{dx^l} (x^2-1)^l
\]

\section{Some results}

\ba
	\frac{1}{1-z} &= \sum_{n=0}^\infty z^n \\
	\frac{z}{(1-z)^2} &= \sum_{n=0}^\infty n z^n \\
	\sin z &=  \sum_{n=0}^\infty  \frac{(-1)^n}{(2n+1)!} z^{2n+1} \\
	\cos z &=  \sum_{n=0}^\infty  \frac{(-1)^n}{(2n)!} z^{2n} \\
\ea

\end{document}


