\documentclass[12pt,twoside]{article}
\usepackage[dvipsnames]{xcolor}
\usepackage{tikz,graphicx,amsmath,amsfonts,amscd,amssymb,mathrsfs, bm,cite,epsfig,epsf,url}
\usepackage[hang,flushmargin]{footmisc}
\usepackage[colorlinks=true,urlcolor=blue,citecolor=blue]{hyperref}
\usepackage{amsthm,multirow,wasysym,appendix}
\usepackage{array,subcaption} 
% \usepackage[small,bf]{caption}
\usepackage{bbm}
\usepackage{pgfplots}
\usetikzlibrary{spy}
\usepgfplotslibrary{external}
\usepgfplotslibrary{fillbetween}
\usetikzlibrary{arrows,automata}
\usepackage{thmtools}
\usepackage{blkarray} 
\usepackage{textcomp}
\usepackage[left=0.8in,right=1.0in,top=1.0in,bottom=1.0in]{geometry}


\usepackage{times}
\usepackage{amsfonts}
\usepackage{amsmath}
\usepackage{latexsym}
\usepackage{color}
\usepackage{graphics}
\usepackage{enumerate}
\usepackage{amstext}
\usepackage{blkarray}
\usepackage{url}
\usepackage{epsfig}
\usepackage{bm}
\usepackage{hyperref}
\hypersetup{
    colorlinks=true,
    linkcolor=blue,
    filecolor=magenta,      
    urlcolor=blue,
}
\usepackage{textcomp}
\usepackage[left=0.8in,right=1.0in,top=1.0in,bottom=1.0in]{geometry}
\usepackage{mathtools}
\usepackage{minted}



%% Probability operators and functions
%
% \def \P{\mathrm{P}}
\def \P{\mathrm{P}}
\def \E{\mathrm{E}}
\def \Var{\mathrm{Var}}
\let\var\Var
\def \Cov {\mathrm{Cov}} \let\cov\Cov
\def \MSE {\mathrm{MSE}} \let\mse\MSE
\def \sgn {\mathrm{sgn}}
\def \R {\mathbb{R}}
\def \C {\mathbb{C}}
\def \N {\mathbb{N}}
\def \Z {\mathbb{Z}}
\def \cV {\mathcal{V}}
\def \cS {\mathcal{S}}

\newcommand{\RR}{\ensuremath{\mathbb{R}}}

\DeclareMathOperator*{\argmin}{arg\,min}
\DeclareMathOperator*{\argmax}{arg\,max}
\newcommand{\red}[1]{\textcolor{red}{#1}}
\newcommand{\blue}[1]{\textcolor{blue}{#1}}
\newcommand{\green}[1]{\textcolor{ForestGreen}{ #1}}
\newcommand{\fuchsia}[1]{\textcolor{RoyalPurple}{ #1}}



%
%% Probability distributions
%
%\def \Bern    {\mathrm{Bern}}
%\def \Binom   {\mathrm{Binom}}
%\def \Exp     {\mathrm{Exp}}
%\def \Geom    {\mathrm{Geom}}
% \def \Norm    {\mathcal{N}}
%\def \Poisson {\mathrm{Poisson}}
%\def \Unif    {\mathrm {U}}
%
\DeclareMathOperator{\Norm}{\mathcal{N}}

\newcommand{\bdb}[1]{\textcolor{red}{#1}}

\newcommand{\ml}[1]{\mathcal{ #1 } }
\newcommand{\wh}[1]{\widehat{ #1 } }
\newcommand{\wt}[1]{\widetilde{ #1 } }
\newcommand{\conj}[1]{\overline{ #1 } }
\newcommand{\rnd}[1]{\tilde{ #1 } }
\newcommand{\rv}[1]{ \rnd{ #1}  }
\newcommand{\rx}{\rnd{ x}  }
\newcommand{\ry}{\rnd{ y}  }
\newcommand{\rz}{\rnd{ z}  }
\newcommand{\ra}{\rnd{ a}  }
\newcommand{\rb}{\rnd{ b}  }
\newcommand{\rpc}{\widetilde{ pc}  }
\newcommand{\rndvec}[1]{\vec{\rnd{#1}}}

\def \cnd {\, | \,}
\def \Id { I }
\def \J {\mathbf{1}\mathbf{1}^T}

\newcommand{\op}[1]{\operatorname{#1}}
\newcommand{\setdef}[2]{ := \keys{ #1 \; | \; #2 } }
%\newcommand{\set}[2]{ \keys{ #1 \; | \; #2 } }
\newcommand{\sign}[1]{\op{sign}\left( #1 \right) }
\newcommand{\trace}[1]{\op{tr}\left( #1 \right) }
\newcommand{\tr}[1]{\op{tr}\left( #1 \right) }
\newcommand{\inv}[1]{\left( #1 \right)^{-1} }
%\newcommand{\abs}[1]{\left| #1 \right|}
\newcommand{\sabs}[1]{| #1 |}
\newcommand{\keys}[1]{\left\{ #1 \right\}}
\newcommand{\sqbr}[1]{\left[ #1 \right]}
\newcommand{\sbrac}[1]{ ( #1 ) }
\newcommand{\brac}[1]{\left( #1 \right) }
\newcommand{\bbrac}[1]{\big( #1 \big) }
\newcommand{\Bbrac}[1]{\Big( #1 \Big)}
\newcommand{\BBbrac}[1]{\BIG( #1 \Big)}
\newcommand{\MAT}[1]{\begin{bmatrix} #1 \end{bmatrix}}
\newcommand{\sMAT}[1]{\left(\begin{smallmatrix} #1 \end{smallmatrix}\right)}
\newcommand{\sMATn}[1]{\begin{smallmatrix} #1 \end{smallmatrix}}
\newcommand{\PROD}[2]{\left \langle #1, #2\right \rangle}
\newcommand{\PRODs}[2]{\langle #1, #2 \rangle}
\newcommand{\der}[2]{\frac{\text{d}#2}{\text{d}#1}}
\newcommand{\pder}[2]{\frac{\partial#2}{\partial#1}}
\newcommand{\derTwo}[2]{\frac{\text{d}^2#2}{\text{d}#1^2}}
\newcommand{\ceil}[1]{\lceil #1 \rceil}
\newcommand{\Imag}[1]{\op{Im}\brac{ #1 }}
\newcommand{\Real}[1]{\op{Re}\brac{ #1 }}
%\newcommand{\norm}[1]{\left|\left| #1 \right|\right| }
\newcommand{\norms}[1]{ \| #1 \|  }
\newcommand{\normProd}[1]{\left|\left| #1 \right|\right| _{\PROD{\cdot}{\cdot}} }
\newcommand{\normTwo}[1]{\left|\left| #1 \right|\right| _{2} }
\newcommand{\normTwos}[1]{ \| #1  \| _{2} }
\newcommand{\normZero}[1]{\left|\left| #1 \right|\right| _{0} }
\newcommand{\normTV}[1]{\left|\left| #1 \right|\right|  _{ \op{TV}  } }% _{\op{c} \ell_1} }
\newcommand{\normOne}[1]{\left|\left| #1 \right|\right| _{1} }
\newcommand{\normOnes}[1]{\| #1 \| _{1} }
\newcommand{\normOneTwo}[1]{\left|\left| #1 \right|\right| _{1,2} }
\newcommand{\normF}[1]{\left|\left| #1 \right|\right| _{\op{F}} }
\newcommand{\normLTwo}[1]{\left|\left| #1 \right|\right| _{\ml{L}_2} }
\newcommand{\normNuc}[1]{\left|\left| #1 \right|\right| _{\ast} }
\newcommand{\normOp}[1]{\left|\left| #1 \right|\right|  }
\newcommand{\normInf}[1]{\left|\left| #1 \right|\right| _{\infty}  }
\newcommand{\proj}[1]{\mathcal{P}_{#1} \, }
\newcommand{\diff}[1]{ \, \text{d}#1 }
\newcommand{\vc}[1]{\boldsymbol{\vec{#1}}}
\newcommand{\rc}[1]{\boldsymbol{#1}}
\newcommand{\vx}{\vec{x}}
\newcommand{\vy}{\vec{y}}
\newcommand{\vz}{\vec{z}}
\newcommand{\vu}{\vec{u}}
\newcommand{\vv}{\vec{v}}
\newcommand{\vb}{\vec{\beta}}
\newcommand{\va}{\vec{\alpha}}
\newcommand{\vaa}{\vec{a}}
\newcommand{\vbb}{\vec{b}}
\newcommand{\vg}{\vec{g}}
\newcommand{\vw}{\vec{w}}
\newcommand{\vh}{\vec{h}}
\newcommand{\vbeta}{\vec{\beta}}
\newcommand{\valpha}{\vec{\alpha}}
\newcommand{\vgamma}{\vec{\gamma}}
\newcommand{\veta}{\vec{\eta}}
\newcommand{\vnu}{\vec{\nu}}
\newcommand{\rw}{\rnd{w}}
\newcommand{\rvnu}{\vc{\nu}}
\newcommand{\rvv}{\rndvec{v}}
\newcommand{\rvw}{\rndvec{w}}
\newcommand{\rvx}{\rndvec{x}}
\newcommand{\rvy}{\rndvec{y}}
\newcommand{\rvz}{\rndvec{z}}
\newcommand{\rvX}{\rndvec{X}}


\newtheorem{theorem}{Theorem}[section]
% \declaretheorem[style=plain,qed=$\square$]{theorem}
\newtheorem{corollary}[theorem]{Corollary}
\newtheorem{definition}[theorem]{Definition}
\newtheorem{lemma}[theorem]{Lemma}
\newtheorem{remark}[theorem]{Remark}
\newtheorem{algorithm}[theorem]{Algorithm}

% \theoremstyle{definition}
%\newtheorem{example}[proof]{Example}
\declaretheorem[style=definition,qed=$\triangle$,sibling=definition]{example}
\declaretheorem[style=definition,qed=$\bigcirc$,sibling=definition]{application}

%
%% Typographic tweaks and miscellaneous
%\newcommand{\sfrac}[2]{\mbox{\small$\displaystyle\frac{#1}{#2}$}}
%\newcommand{\suchthat}{\kern0.1em{:}\kern0.3em}
%\newcommand{\qqquad}{\kern3em}
%\newcommand{\cond}{\,|\,}
%\def\Matlab{\textsc{Matlab}}
%\newcommand{\displayskip}[1]{\abovedisplayskip #1\belowdisplayskip #1}
%\newcommand{\term}[1]{\emph{#1}}
%\renewcommand{\implies}{\;\Rightarrow\;}

% My macros

\def\Kset{\mathbb{K}}
\def\Nset{\mathbb{N}}
\def\Qset{\mathbb{Q}}
\def\Rset{\mathbb{R}}
\def\Sset{\mathbb{S}}
\def\Zset{\mathbb{Z}}
\def\squareforqed{\hbox{\rlap{$\sqcap$}$\sqcup$}}
\def\qed{\ifmmode\squareforqed\else{\unskip\nobreak\hfil
\penalty50\hskip1em\null\nobreak\hfil\squareforqed
\parfillskip=0pt\finalhyphendemerits=0\endgraf}\fi}

%\DeclareMathOperator*{\E}{\rm E}
%\DeclareMathOperator*{\argmax}{\rm argmax}
%\DeclareMathOperator*{\argmin}{\rm argmin}
%\DeclareMathOperator{\sgn}{sign}
\DeclareMathOperator{\supp}{supp}
\DeclareMathOperator{\last}{last}
%\DeclareMathOperator{\sign}{\sgn}
\DeclareMathOperator{\diag}{diag}
\providecommand{\abs}[1]{\lvert#1\rvert}
\providecommand{\norm}[1]{\lVert#1\rVert}
\def\vcdim{\textnormal{VCdim}}
\DeclareMathOperator*{\B}{\textbf{B}}

%\DeclarePairedDelimiter\ceil{\lceil}{\rceil}
%\DeclarePairedDelimiter\floor{\lfloor}{\rfloor}

\newcommand{\cX}{{\mathcal X}}
\newcommand{\cY}{{\mathcal Y}}
\newcommand{\cA}{{\mathcal A}}
\newcommand{\ignore}[1]{}
\newcommand{\ba}{\[\begin{aligned}}
\newcommand{\ea}{\end{aligned}\]}
\newcommand{\bi}{\begin{itemize}}
\newcommand{\ei}{\end{itemize}}
\newcommand{\be}{\begin{enumerate}}
\newcommand{\ee}{\end{enumerate}}
\newcommand{\bd}{\begin{description}}
\newcommand{\ed}{\end{description}}
\newcommand{\h}{\widehat}
\newcommand{\e}{\epsilon}
\newcommand{\mat}[1]{{\mathbf #1}}
%\newcommand{\R}{\mat{R}}
\newcommand{\0}{\mat{0}}
\newcommand{\M}{\mat{M}}

\newcommand{\D}{\mat{D}}
\renewcommand{\r}{\mat{r}}
\newcommand{\x}{\mat{x}}
\renewcommand{\u}{\mat{u}}
\renewcommand{\v}{\mat{v}}
\newcommand{\w}{\mat{w}}
\renewcommand{\H}{\text{0}}
\newcommand{\T}{\text{1}}
%\newcommand{\set}[1]{\{#1\}}
\newcommand{\xxi}{{\boldsymbol \xi}}
\newcommand{\ssigma}{{\boldsymbol \sigma}}
\newcommand{\Alpha}{{\boldsymbol \alpha}}
\newcommand{\tts}{\tt \small}
\newcommand{\hint}{\emph{hint}}
\newcommand{\matr}[1]{\bm{#1}}     % ISO complying version
\newcommand{\vect}[1]{\bm{#1}} % vectors

%\newcommand{\Var}{\mathrm{Var}}
%\newcommand{\Cov}{\mathrm{Cov}}

% New commands
\newcommand{\SP}{\mathbf{S}_{+}^n}
\newcommand{\Proj}{\mathcal{P}_{\mathcal{S}}}
%\DeclarePairedDelimiterX{\inp}[2]{\langle}{\rangle}{#1, #2}



\begin{document}

\noindent Professor Rio\\
EN.585.615.81.SP21 Mathematical Methods\\
Mid-term Exam\\
Johns Hopkins University\\
Student: Yves Greatti\\\

\section*{Question 7}
\[
	x^2 \frac{d^2 y}{dx^2} + x \frac{dy}{d} - y = x \, , y(e) = 0 \, , y'(e) = 2
\]

\be 
\item [a.]
This is Euler differential equation, and we make the change of variable $x=e^t$ or $t=\ln(x)$.
Then
\ba
	\frac{dy}{dx} &= \frac{dy}{dt} \frac{dt}{dx} = \frac{dy}{dt} \frac{d \ln x}{dx} =  \frac{dy}{dt} \frac{1}{x} = \frac{1}{x}   \frac{dy}{dt} \\ 
	x \frac{dy}{dx}  &= \frac{dy}{dt} \\
\ea
And since this is a Legendre ODE with $\alpha=1$ and $\beta=0$, we can use the expression for the second derivative 
$(\alpha x + \beta)^2 \frac{d^2y}{dx^2} = \alpha^2 \frac{d}{d t} [\frac{d}{dt} - 1] y$.
With $\alpha=1$ and $\beta=0$, we have: $ \frac{d^2y}{dx^2} =  \frac{d^2y}{ t^2} - \frac{dy}{dt}$. \\ \hfill \break

Substitute into the above equation yields:
\ba
	( \frac{d^2y}{dt^2} - \frac{dy}{dt}) + \frac{dy}{dt} - y &= e^t \\
	 \frac{d^2y}{dt^2} - y &= e^t \\
\ea

\item [b.]
The homogeneous equation is
\[
	 \frac{d^2y}{dt^2} - y = 0
\]

Assume a solution of the form $y(t) = A e^{\lambda t}$ gives the characteristic equation $\lambda^2 - 1 = 0$ which has for roots $\lambda = \pm1$ and gives
for solution $y(t) = c_1 e^t + c_2 e^{-t}$.

\item [c.]
The ODE to solve is:
\[
	 \frac{d^2y}{dt^2} - y = 0
\]
It is in standard form and it is defined at any point $t$, it is analytic, thus we take as solution $y(t) = \sum_{t=0}^\infty a_n t^n$.
So:
\ba
	y'(t)  &=  \sum_{t=0}^\infty n a_n t^{n-1} \\
	y''(t) &=  \sum_{t=0}^\infty n (n-1) a_n t^{n-2} \\
	\text{by reindexing}\\
	y''(t) &=  \sum_{t=-2}^\infty (n+2) (n+1) a_{n+2} t^n \\
	y''(t) &=  \sum_{t=0}^\infty (n+2) (n+1) a_{n+2} t^n \\
\ea
Substitute into the ODE gives:
\ba
	  \sum_{t=0}^\infty (n+2) (n+1) a_{n+2} t^n -  \sum_{t=0}^\infty a_n t^n &= 0\\
	  \sum_{t=0}^\infty [ (n+2) (n+1) a_{n+2} - a_n] t^n & = 0\\
\ea
or
\ba
a_{n+2} &= \frac{1}{(n+2) (n+1)} a_n \\
a_n &= \frac{1}{ n (n-1)} a_{n-2} \\
\ea
Take $a_0 = a_1 = 1$ and we generate the coefficients:
\be
	\item[.] $n = 2$ then $a_2 = \frac{1}{2 \cdot 1} a_0 = \frac{1}{2 \cdot 1} = \frac{1}{2!}$
	\item[.] $n = 3$ then $a_3 = \frac{1}{3 \cdot 2} a_1 = \frac{1}{3 \cdot 2} = \frac{1}{3!}$
	\item[.] $n = 4$ then $a_4 = \frac{1}{4 \cdot 3} a_2 = \frac{1}{4 \cdot 3 \cdot 2 \cdot 1} = \frac{1}{4!}$
	\item[$\vdots$]
	\item[.] $a_n = \frac{1}{ n (n-1)} a_{n-2} = \cdots = \frac{1}{n!}$
\ee
The first solution we obtain is: $y_1(t) =  \sum_{t=0}^\infty a_n t^n =  \sum_{t=0}^\infty \frac{t^n}{n!} = e^t$.
Secondly, if we set $a_0=1$ and choose $a_1=-1$, then we obtain a second independent solution:
\be
	\item[.] $n = 2$ then $a_2 = \frac{1}{2 \cdot 1} a_0 = \frac{1}{2 \cdot 1} = \frac{1}{2!}$
	\item[.] $n = 3$ then $a_3 = \frac{1}{3 \cdot 2} a_1 = - \frac{1}{3 \cdot 2} = \frac{-1}{3!}$
	\item[.] $n = 4$ then $a_4 = \frac{1}{4 \cdot 3} a_2 = \frac{1}{4 \cdot 3 \cdot 2 \cdot 1} = \frac{1}{4!}$
	\item[.] $n = 5$ then $a_5 = \frac{1}{5 \cdot 4} a_3 = \frac{-1}{5 \cdot 4 \cdot 3 \cdot 2 \cdot 1} = \frac{-1}{5!}$
	\item[$\vdots$]
	\item[.] $a_n = \frac{1}{ n (n-1)} a_{n-2} = \cdots = \frac{(-1)^n}{n!}$
\ee
We have the second solution: $y_2(t) =  \sum_{t=0}^\infty a_n t^n =  \sum_{t=0}^\infty \frac{(-t)^n}{n!}$, 
recognizing the last series as $e^{-t}$, we can write the general solution of the homogeneous equation as 
\[
	y_H(t) = c_1 e^t + c_2 e^{-t}
\]
which is the solution we found in question b.

\item[d.]
The differential equation to solve is
\[
	 \frac{d^2y}{dt^2} - y = e^t
\]
Next we use the variation of parameters method, we are looking for a solution $y_p(t) = k_1(t) e^t + k_2(t) e^{-t}$.
We solve for derivatives of k's a system of two equations:
\begin{align*}
&
	\begin{cases}
		k'_1 e^t + k'_2 e^{-t} & = 0 \\
		k'_1 e^t - k'_2 e^{-t} & = e^t \\
	\end{cases}
\end{align*}
Multiplying through by $e^t$ gives:
\begin{align*}
&
	\begin{cases}
		k'_1 e^{2t} + k'_2 & = 0 \\
		k'_1 e^{2t} - k'_2  & = e^{2t} \\
	\end{cases}
\end{align*}

Adding first equation to second yields $2 k'_1 e^{2t} = e^{2t}$ or $k'_1 = \frac{1}{2}$ and $k_1 = \frac{t}{2}$.
Substitute
\ba
	k'_2 &= - k'_1 e^{2t} \\
		&= - \frac{1}{2} e^{2t}  \\
	\text{integrating} \\
	k_2 &= - \frac{e^{2t}}{4} \\
\ea
Therefore:
\ba
	y_p(t) 	&= k_1(t) e^t + k_2(t) e^{-t} \\
			&=  \frac{t}{2} e^t - \frac{e^{2t}}{4} e^{-t} \\
			&=  \frac{t}{2} e^t - \frac{e^t}{4} \\
			&= \frac{e^t}{2} (t - \frac{1}{2}) \\
\ea

\item [e.] The general solution is: $y(t) =  y_H(t) + y_p(t) = c_1 e^t + c_2 e^{-t} + \frac{e^t}{2} (t - \frac{1}{2})$, simplifying the constants,
we can rewrite the general solution as $y(t) =  c_1 e^t + c_2 e^{-t} + \frac{t}{2} e^t$.
Plugging back $x=e^t$ or $t=\ln(x)$ gives 
\[
	y(x) = c_1 x + \frac{c_2}{x} + \frac{x \ln x}{ 2}
\]

\item [f.]
The total solution is
\ba
	y(x) 	&= c_1 x + \frac{c_2}{x} + \frac{x \ln x}{2} \\
	y'(x) 	&= c_1 x - \frac{c_2}{x^2} + \frac{1}{2} (1 + \ln x) \\
\ea
And the initial conditions are $y(e) = 0 \, , y'(e) = 2$, plugging back these into the previous equations gives 
\ba
	\begin{cases}
	y(e) = c_1 e + \frac{c_2}{e} + \frac{e \ln e} {2} = 0 \\
	y'(e) = c_1 - \frac{c_2}{e^2} + \frac{1} {2} (1 +  \ln e)= 2 \\
	\end{cases} \\	
\ea
\ba
	\Rightarrow 
	\begin{cases}
	c_1 e + c_2 e^{-1}  = - \frac{e}{2} \\
	c_1 - c_2 e^{-2} = 1 \\
	\end{cases} \\	
\ea
\ba
	\Rightarrow 
	\begin{cases}
	c_1 e^2 + c_2  = - \frac{e^2}{2} \\
	c_1 - c_2 e^{-2} = 1 \\
	\end{cases} \\	
\ea
Adding equation (1) to equation (2) leads to $2 c_1 = e^2 - \frac{e^2}{2} =  \frac{e^2}{2}$,  $c_1=\frac{1}{4}$,$c_2 = e^2 (c_1-1) = \frac{3}{4} e^2$. 
Reporting these constants into the expression of the total solution gives:
\[
	y(x) = \frac{1}{4} x -  \frac{3}{4} e^2 \frac{1}{x} + \frac{x \ln x}{2} 
\]
\[
	y(x) = \frac{x^2 + 2 x^2 \ln(x) - 3 e^2} { 4 x} 
\]

\ee
\end{document}
