\documentclass[12pt,twoside]{article}
\usepackage[dvipsnames]{xcolor}
\usepackage{tikz,graphicx,amsmath,amsfonts,amscd,amssymb,mathrsfs, bm,cite,epsfig,epsf,url}
\usepackage[hang,flushmargin]{footmisc}
\usepackage[colorlinks=true,urlcolor=blue,citecolor=blue]{hyperref}
\usepackage{amsthm,multirow,wasysym,appendix}
\usepackage{array,subcaption} 
% \usepackage[small,bf]{caption}
\usepackage{bbm}
\usepackage{pgfplots}
\usetikzlibrary{spy}
\usepgfplotslibrary{external}
\usepgfplotslibrary{fillbetween}
\usetikzlibrary{arrows,automata}
\usepackage{thmtools}
\usepackage{blkarray} 
\usepackage{textcomp}
\usepackage[left=0.8in,right=1.0in,top=1.0in,bottom=1.0in]{geometry}


\usepackage{times}
\usepackage{amsfonts}
\usepackage{amsmath}
\usepackage{latexsym}
\usepackage{color}
\usepackage{graphics}
\usepackage{enumerate}
\usepackage{amstext}
\usepackage{blkarray}
\usepackage{url}
\usepackage{epsfig}
\usepackage{bm}
\usepackage{hyperref}
\hypersetup{
    colorlinks=true,
    linkcolor=blue,
    filecolor=magenta,      
    urlcolor=blue,
}
\usepackage{textcomp}
\usepackage[left=0.8in,right=1.0in,top=1.0in,bottom=1.0in]{geometry}
\usepackage{mathtools}
\usepackage{minted}



\input{macros}

\begin{document}

\noindent Professor Rio\\
EN.585.615.81.SP21 Mathematical Methods\\
Midterm Exam\\
Johns Hopkins University\\
Student: Yves Greatti\\\

\section*{Question 1}

\be
\item [a.]
Graph of the function attached in a separate pdf.

\item [b.]
Since we have made the function $f(x)$ even using an even extension, all the $b_k$ coefficients in its Fourier series are zero.
With a period $L=4$, we determine the remaining coefficients $a_k$:

\[
	a_k = \frac{2}{4} \int_{-2}^2 x \cos{( \frac{2 k \pi x}{4} )} dx
\]
And since $f$ is even now
\ba
	a_k &=	\frac{4}{4} \int_0^2 x\cos{( \frac{2 k \pi x}{4} )} dx \\
		&=  \int_0^2 x \cos{( \frac{k \pi x}{2} )} dx \\
\ea
Using integration by parts, for $k > 0$:
\ba
	a_k	&=	\frac{2}{k \pi} [x \sin(\frac{k \pi x}{2}) ]_0^2 - \frac{2}{k \pi} \int_0^2  \sin(\frac{k \pi x}{2}) dx \\
		&= 0  - \frac{2}{k \pi} (- \frac{2}{k \pi}) [ \cos \frac{k \pi x}{2})]_0^2 \\
		&= \frac{4}{(k \pi)^2)} [\cos (k \pi) - \cos(0)] \\
		&=  \frac{4}{(k \pi)^2} [(-1)^k - 1] \\
\ea
Then 
\[
 a_k =
   \begin{dcases}
   - \frac{8} {(k \pi)^2} ~ \text{for odd } k\\
     0 ~ \text{for even } k\\
   \end{dcases}
\]
And $a_0 = \frac{2}{4} \int_{-2}^2 x dx = \frac{4}{4}  \int_0^2 x dx  = \frac{1}{2} [x^2]_0^2 = 2$.
With the coefficients $a_k$ determined, we obtain the Fourier series for $f(x)$:

\ba
	f(x) &= \frac{2}{2} - \sum_{k=1}^\infty \frac{8} {(k \pi)^2} \cos( \frac{2 k \pi x} {4}) \, k \text{ odd} \\
	x	&= 1 - \frac{8} {\pi^2}   \sum_{k=0}^\infty \frac{1}{(2k+1)^2} \cos( \frac{(2k+1) \pi x} {2}) \\
\ea

\item [c.]
Applying Parseval's identity for Fourier series and using the result of part b.:

\ba
	\frac{1}{4} \int_{-2}^2 x^2 dx			&= \frac{2^2}{4} + \frac{1}{2}  \sum_{k=1}^\infty (a_k^2 + 0)  \: k \text{ odd} \\
	\frac{2}{4} \int_0^2  x^2 dx				&= 1 +  \frac{1}{2}   \sum_{k=0}^\infty (\frac{8} {(2k+1)^2 \pi^2})^2 \\
	\frac{1}{2} [\frac{x^3}{3}]_0^2			&= 1 +  \frac{1}{2}   \cdot  \frac{64}{\pi^4}   \sum_{k=0}^\infty  \frac{1} {(2k+1)^4} \\
	\frac{4}{3}	- 1						&= \frac{32}{\pi^4}   \sum_{k=0}^\infty  \frac{1} {(2k+1)^4} \\
	 \sum_{k=0}^\infty  \frac{1} {(2k+1)^4}	&= \frac{\pi^4}{32} \cdot \frac{1}{3} \\
\ea
Therefore
\[
	 \sum_{n=0}^\infty  \frac{1} {(2n+1)^4} = \frac{\pi^4}{96}
\]
\ee

\section*{Question 2}
\be 
\item [a.]
Graph of the function attached in a separate pdf.

\item [b.]
\[
	f(t) = A \bigg [ H(t) - H(t-\tau) \bigg ]
\]

\item [c.]

\[
	\tilde{f}(w) = F\{f(t)\} =  \frac{1}{\sqrt{2 \pi}} \int_{-\infty}^{\infty} f(t) e^{-i w t} \, dt
\]
Since $f(t)=0$ for $t\ge0$ or $t\le\tau$:

\ba
	\tilde{f}(w)	&= \frac{1}{\sqrt{2 \pi}} \int_0^\tau A \cdot e^{-i w t} \, dt \\
			&= \frac{A}{\sqrt{2 \pi}}  (\frac{1}{-iw}) [e^{-i w t}]_0^\tau \\
			&= \frac{i A}{w \sqrt{2 \pi}} (e^{-i w \tau} - 1) \\
			&= \frac{i A}{w \sqrt{2 \pi}}  e^{-i w \frac{\tau}{2}} (e^{-i w \frac{\tau}{2}} - e^{i w \frac{\tau}{2}}) \\
\ea
From Euler identity: 
\[
	e^{-i w \frac{\tau}{2}} - e^{i w \frac{\tau}{2}} = -2 i \sin  w \frac{\tau}{2}
\]
Therefore
\ba
	\tilde{f}(w)	&= \frac{2A}{w \sqrt{2 \pi}}  e^{-i w \frac{\tau}{2}} \sin  w \frac{\tau}{2} \\
			&= \sqrt{\frac{2}{\pi}} \frac{A}{w} e^{-i w \frac{\tau}{2}} \sin  w \frac{\tau}{2} \\
			&= A  \sqrt{\frac{2}{\pi}}  e^{-i w \frac{\tau}{2}}  \frac{\tau}{2} \frac{ \sin(w \frac{\tau}{2}) } {w \frac{\tau}{2}} \\
			&= \frac{A}  {\sqrt{2 \pi}} \tau e^{-i w \frac{\tau}{2}} \text{sinc}(w \frac{\tau}{2}) \\
\ea

\item [d.]
Let $A=\frac{1}{\tau}$ then substituting in $f(t)$	from part c., gives:
\[
	F\{ \lim_{\tau \rightarrow 0} f(t) \} =  \lim_{\tau \rightarrow 0}  F\{ f(t) \} =  \lim_{\tau \rightarrow 0}   \frac{1}  {\sqrt{2 \pi}}  e^{-i w \frac{\tau}{2}} \frac{ \sin(w \frac{\tau}{2}) } {w \frac{\tau}{2}}
\]
\ba
	 \lim_{\theta \rightarrow 0} \frac{\sin(\theta)} {\theta} &= 1 ~ \text{ by Hospitals rule} \\
	  \lim_{\tau \rightarrow 0}  e^{-i w \frac{\tau}{2}} 		&=   \lim_{\tau \rightarrow 0} e^0 = 1 \\
\ea
Therefore
\[
	F\{ \lim_{\tau \rightarrow 0} f(t) \} =  \frac{1}  {\sqrt{2 \pi}} 
\]

\item [e.]
The Fourier transform of $f(t)$ as $\tau  \rightarrow 0$ is the Fourier transform of a $\delta$-function as we can expect since 
as   $\tau  \rightarrow 0$  the rectangular function $f(t)$  $\rightarrow$ a  $\delta$-function.
\ee

\section*{Question 3}

\be

\item [a.]
By definition, the Laplace transform of $g(t) = \sin(5t)$ is:
\[
	\bar{g}(s) = L\{g(t)\} = \int_0^\infty \sin(5t) e^{-st} dt = \lim_{L  \rightarrow \infty}  \int_0^L \sin(5t) e^{-st} dt
\]

First compute $\int e^{-st} \sin at \, dt$, using integration by parts with $u=\sin at$, $u' = a \cos at$, $v'=e^{-st}$, $v=-\frac{1}{s} e^{-st}$:
\begin{equation}
 \int e^{-st} \sin(at) \, dt = -\frac{1}{s} e^{-st} \sin(at) + \frac{a}{s} \int e^{-st} \cos(at) dt 
\end{equation}

Next compute $\int e^{-st} \cos(at) \, dt$, again, using integration by parts with $u=\cos at$, $u' = -a \sin at$, $v'=e^{-st}$, $v=-\frac{1}{s} e^{-st}$:
\[
	\int e^{-st} \cos(at) \, dt =  -\frac{1}{s} e^{-st} \cos(at) - \frac{a}{s} \int e^{-st} \sin(at) \, dt 
\]
Substituting into (1):
\ba
	 \int e^{-st} \sin(at) \, dt 	&= -\frac{1}{s} e^{-st} \sin(at) + \frac{a}{s} \bigg (  -\frac{1}{s} e^{-st} \cos(at) - \frac{a}{s} \int e^{-st} \sin(at) \, dt \bigg ) \\
	 					&= -\frac{1}{s} e^{-st} \sin(at) - \frac{a}{s^2} e^{-st} \cos(at) + \frac{a}{s^2} \int e^{-st} \sin(at) \, dt \\
\ea
thus
\[
	(1 + \frac{a^2}{s^2})  \int e^{-st} \sin(at) \, dt = - e^{-st} ( \frac{1}{s} \sin(at) + \frac{a}{s^2} \cos(at))
\]
Evaluating at $t=0$ and $t \rightarrow \infty$:
\ba
	(1 + \frac{a^2}{s^2}) L\{ \sin(at)\}	&= \lim_{L \rightarrow \infty} \bigg [ - e^{-st} ( \frac{1}{s} \sin(at) + \frac{a}{s^2} \cos(at)) \bigg ]_0^L \\
							&= 0 - (-1 (\frac{1}{s} \cdot 0 + \frac{a}{s^2} \cdot 1)) \\
							&= \frac{a}{s^2} 
\ea
Therefore
\ba
	L\{ \sin(at)\}	&= \frac{a}{s^2}  (1 + \frac{a^2}{s^2})^{-1} \\
				&= \frac{a}{a^2 + s^2} \\
\ea
Set $a=5$ and 
\[
	L\{g(t)\} = L\{ \sin(5t)\} = \frac{5}{s^2 + 25}
\]

\item [b.]
 From the book, one property of the Laplace transform is $L[t^n f(t)] = (-1)^n \frac{d^n \bar{f}(s)}{ds^n}$ for $n=1,2,3,\cdots$,
take $n=1$, $L[t  f(t)] = - \frac{d\bar{f}(s)}{ds}$.
Set $f(t) = t \sin(5t)$ and from part b, $L\{ \sin(5t)\} = \frac{5}{s^2 + 25}$, therefore:
\ba
	L\{t \sin(5t)\}	&=  - \frac{d}{ds} \bigg( \frac{5}{s^2 + 25} \bigg ) \\
				&= -5 \frac{d}{ds} \bigg( \frac{1}{s^2 + 25} \bigg ) \\
				&= -5 \bigg ( \frac{-2s} {(s^2+25)^2} \bigg ) \\
				&= \frac{10 s}  {(s^2+25)^2} \\
\ea

\item [c.]
By definition $(f*g)(t) = \int_0^t \tau e^{-(t-\tau)} d\tau = e^{-t} \int_0^t \tau e^\tau d\tau$. Using integration by parts:
\ba
	\int_0^t \tau e^\tau d\tau 	&=  [\tau e^\tau]_0^t - \int_0^t e^\tau d\tau \\
						&= t e^t -  [e^\tau]_0^t  \\
						&= t e^t - (e^t - 1) \\
						&= e^t (t-1) + 1
\ea
And 
\[
	(f*g)(t) = e^{-t} \bigg[  e^{-t} (t-1) + 1 \bigg ] = e^{-t} + t - 1
\]


From $L\{(f*g)(t)\} = \bar{f}(s) \cdot \bar{g}(s)$, we have:
\ba
	\bar{f}(s) \cdot \bar{g}(s)	&=	\frac{1}{s^2} \cdot \frac{1}{s+1} \\
						&= 	\frac{1-s}{s^2+1} + \frac{1}{s+1} \\
						&= 	\frac{1}{s^2} - \frac{1}{s} + \frac{1}{s+1} \\
\ea
Therefore
\ba
	(f*g)(t) 	&=	L^{-1}\{L\{(f*g)(t)\}\} = L^{-1} \{ \frac{1}{s^2} - \frac{1}{s} + \frac{1}{s+1} \} \\
			&=	L^{-1}\{\frac{1}{s^2} \} - L^{-1}\{\ \frac{1}{s} \}  + L^{-1}\{\ \frac{1}{s+1} \} \\
			&= 	t -1 + e^{-t} \\
			&=  	e^{-t} + t - 1 \\
\ea

\ee

\section*{Question 4}
\[
	y'' + 4 y' - 5 y= \delta(t-1) \: y(0) = 0 \: y'(0) = 3
\]

\be
\item [a.]
Taking the Laplace transform on both sides of the equation gives:
\ba
	s^2 \tilde{y}(s) -s y(0) - y'(0) + 4 [ s  \tilde{y}(s) - y(0) ] -5  \tilde{y}(s) &= e^{-s} \\
	s^2 \tilde{y}(s) -s \cdot 0 -3 + 4 [ s  \tilde{y}(s) - 0] -5  \tilde{y}(s) &= e^{-s} \\
	s^2 \tilde{y}(s) + 4 s  \tilde{y}(s) -5  \tilde{y}(s) &= e^{-s} + 3
\ea
Combining the terms: $ (s^2 + 4 s  -5) \tilde{y}(s) = 3 +  e^{-s}$.
Therefore
\[
	  \tilde{y}(s) = \frac{3 + e^{-s}} {s^2 + 4 s  -5}
\]

\item [b.]
The roots of $s^2 + 4 s  -5 = 0$ are $-5$ and $1$, so we can rewrite $\tilde{y}(s)$ as $ \tilde{y}(s) = \frac{3}{(s-1) (s+5)} + \frac{e^{-s}} {(s-1) (s+5)}$
Computing the fraction expansion:
\ba
	 \frac{1}{(s-1) (s+5)}	&= \frac{A}{s-1} + \frac{B}{s+5} \\
	 				&= \frac{(A+B) s + 5A - B}{(s-1) (s+5)} \\
\ea
Equating the powers of $s$ on each side of the previous equation:
\ba
	s^1: A + B &= 0 \\
	s^0 : 5 A - B &= 1 \\
\ea
gives $A=\frac{1}{6}$ and $B=-\frac{1}{6}$.
Thus
\[
	\frac{1}{(s-1) (s+5)} = \frac{1}{6} \bigg (\frac{1}{s-1} - \frac{1}{s+5} \bigg )
\]

So
\ba
	\tilde{y}(s)	&= 3 \bigg [  \frac{1}{6} (\frac{1}{s-1} - \frac{1}{s+5})  \bigg ] +  \frac{1}{6} (\frac{e^{-s}}{s-1} - \frac{e^{-s}}{s+5})  \\
			&= \frac{1}{2}  (\frac{1}{s-1} - \frac{1}{s+5}) +  \frac{1}{6} (\frac{e^{-s}}{s-1} - \frac{e^{-s}}{s+5})  \\
\ea

\item [c.]

$y(t) = L^{-1} \{ \tilde{y}(s)	\}$ and from part b:


\[
	y(t) = \frac{1}{2} \bigg [ L^{-1} \{ \frac{1}{s-1} \} - L^{-1} \{ \frac{1}{s+5} \}   \bigg ] +  \frac{1}{6}  \bigg  [ L^{-1} \{ \frac{e^{-s}}{s-1} \} - L^{-1} \{ \frac{e^{-s}}{s+5} \}   \bigg ] 
\]
 $ L^{-1} \{ \frac{1}{s-1} \} = e^t$, $ L^{-1} \{ \frac{1}{s+5}\} = e^{-5t}$, and using the shift theorem:
 \[
 	L\{ f(t-t_0) H(t-t_0) \} = e^{-st_0} F(s) \: \:  f(t-t_0) H(t-t_0)= L^{-1} \{ e^{-st_0} F(s) \}
 \]
So for $t_0=1$
\ba
	 L^{-1} \{ \frac{e^{-s}}{s-1} \}	&= e^{(t-1)} H(t-1) \\
	 L^{-1} \{ \frac{e^{-s}}{s+5} \}	&= e^{-5 (t-1)} H(t-1) \\
\ea
Plugging back these into $y(t)$ yields:
\ba
	y(t)	&= \frac{1}{2} \bigg [ e^t - e^{-5t}  \bigg ] +  \frac{1}{6}  \bigg  [  e^{(t-1)} H(t-1) -  e^{-5 (t-1)} H(t-1)  \bigg ] \\
		&= \frac{1}{2} \bigg [ e^t - e^{-5t}  \bigg ] +  \frac{1}{6}  \bigg (  e^{(t-1)} -  e^{-5 (t-1)}  \bigg ) H(t-1)  \\
		&= \frac{e^t}{2}   \bigg (  1 + \frac{1}{3e} H(t-1)  \bigg ) - \frac{e^{-5t}}{2} (1 + \frac{1}{3 } e^5 H(t-1)  \bigg ) \\
		&= \frac{1}{2} (e^t-e^{-5t}) + \frac{1}{6} (e^{t-1}-e^{-5(t-1)})  H(t-1) \\
\ea

\ee

\section*{Question 5}

\be
\item [a.]
Let rate $r=10\text{ min}^{-1}$, the rate of change of A is equal to how much of A goes to B at rate $r$, how much of A goes to C at rate $r$,
and how much from B goes to A at rate $r$, and  how much from C goes to A at rate $r$. The transport dynamics are the same for B and C
, so the system looks like:
\ba
	\frac{d A}{dt}	&= -r A -r A + r B + r C \\
	\frac{d A}{dt}	&= -2 r A + r B + r C \\
	\frac{d A}{dt}	&= -20 A + 10 B + 10 C ~ \text{ with } A(0) = 20\\
\ea
Similarly
\ba
	\frac{d B}{dt}	&= r A -r B -r B + r C \\
	\frac{d B}{dt}	&= r A - 2 r B + r C \\
	\frac{d B}{dt}	&= 10 A - 20 B + 10 C ~ \text{ with } B(0) = 0\\
\ea
And
\ba
	\frac{d C}{dt}	&= r A + r B -r C - r C \\
	\frac{d C}{dt}	&= r A + r B - 2 r C \\
	\frac{d C}{dt}	&= 10 A + 10 B - 20 C ~ \text{ with } C(0) = 0\\
\ea

\item [b.]
We take the Laplace transforms of the differential equations which gives:
\ba
	s \tilde{A}(s) - A(0)		&= -20 \tilde{A}(s) + 10 \tilde{B}(s) + 10 \tilde{C}(s) \\
	(s+20) \tilde{A}(s) 	 -10 \tilde{B}(s)  -10 \tilde{C}(s) 	&= 20\\
	s \tilde{B}(s) - B(0)		&= 10 \tilde{A}(s) -20 \tilde{B}(s) + 10 \tilde{C}(s) \\
	10 \tilde{A}(s)  - (s+20) \tilde{B}  + 10 \tilde{C}(s) 	&= 0 \\
	s \tilde{C}(s) - C(0)		&= 10 \tilde{A}(s) +10 \tilde{B}(s) -20 \tilde{C}(s) \\
	10 \tilde{A}(s) + 10 \tilde{B}(s) - (s + 20) \tilde{C} 	&= 0 \\
\ea

\item [c.]
We write the equations in matrix form:

\ba
\MAT{
        s+20 & -10  & -10  \\
        10 & -(s+20)  & 10  \\
        10 & 10 & -(s+20)  \\
} 
\MAT{
        \tilde{A}(s) \\
        \tilde{B}(s)   \\
        \tilde{C}(s)  \\
}
 &= 
\MAT{
        20  \\
        0   \\
        0 \\
}
\ea

The determinant of the system is:
\ba
D = \begin{vmatrix}
        s+20 & -10  & -10  \\
        10 & -(s+20)  & 10  \\
        10 & 10 & -(s+20)  \\
\end{vmatrix} &= s^3 + 60 s^2 + 900 s \\
&= s (s^2 + 60 s + 900) \\
&= s (s+30)^2 \\
\ea
Solving using Cramer's rule gives:

\ba
  \tilde{A}(s)  &= \frac{ \begin{vmatrix}
        20 & -10  & -10  \\
        0 & -(s+20)  & 10  \\
        0 & 10 & -(s+20)  \\
\end{vmatrix} } {D} \\
	&= 20  \frac{ \begin{vmatrix}
        -(s+20)  & 10  \\
        10 & -(s+20)  \\
\end{vmatrix} } {D} \\
&= \frac{20 ((s+20)^2 - 100)} {s (s+30)^2} \\
&= \frac{20 (s^2 + 40 s + 300)} {s (s+30)^2} \\
&= \frac{20 (s+10) (s+30)} {s (s+30)^2} \\
&= \frac{20 (s+10)} {s (s+30)} \\
&= \frac{20}{3} [ \frac{1}{s} + \frac{2}{s+30} ] \\
\ea
Taking the inverse Laplace transform using the table:
\[
	A(t) = \frac{20}{3} (1 + 2 e^{-30t})
\]

\ba
  \tilde{B}(s)  &= \frac{ \begin{vmatrix}
        s+20 & 20  & -10  \\
        10 & 0 & 10  \\
        10 & 0 & -(s+20)  \\
\end{vmatrix} } {D} \\
	&= \frac{200 s + 6000} {s (s+30)^2} \\
	&= \frac{200 (s + 30)} {s (s+30)^2} \\
	&= \frac{200} {s (s+30)} \\
	&= \frac{200} {30} \bigg ( \frac{1}{s} - \frac{1}{s+30} \bigg )\\
\ea

Taking the inverse Laplace transform using the table:
\ba
	B(t) 	&= \frac{200}{30} (1 - e^{-30t}) \\
		&= \frac{20}{3} (1 - e^{-30t}) \\
\ea

\ba
  \tilde{C}(s)  &= \frac{ \begin{vmatrix}
        s+20 & -10 & 20 \\
        10 & -(s+20) & 0  \\
        10 & 10 & 0  \\
\end{vmatrix} } {D} \\
	&= \frac{200 s + 6000} {s (s+30)^2} \\
	&= \frac{200 (s + 30)} {s (s+30)^2} \\
	&= \frac{200} {s (s+30)} \\
	&= \frac{200} {30} \bigg ( \frac{1}{s} - \frac{1}{s+30} \bigg )\\
\ea
Taking the inverse Laplace transform using the table:
\[
	C(t) 	= \frac{20}{3} (1 - e^{-30t}) = B(t)
\]
	

\item [d.]
From part c:
\[
	A(t) = \frac{20}{3} (1 + 2 e^{-30t})
\]
\[
	B(t) = C(t) =  \frac{20}{3} (1 - e^{-30t})
\]

\item [e.]
$\lim_{t \rightarrow \infty} e^{-30t} = 0$ therefore as $t \rightarrow \infty$, $\lim_{t \rightarrow \infty}  A(t) = \lim_{t \rightarrow \infty}  B(t)  = \lim_{t \rightarrow \infty}  C(t) = \frac{20}{3}$ which is the equilibrium state of this system
when t goes to infinity. The water has spread out uniformly among the three reservoirs.

\ee

\section*{Question 6}
\[
	L_0(x) = 1 ~ \text{ and } L_n(x) = \frac{e^x}{n!} \frac{d^n}{dx^n}(x^n e^{-x}) ~ n = 1,2,\cdots
\]

Applying the recurrence relationship
\newpage
$n=1$
\ba
	L_1(x) 				&= \frac{e^x}{1!}  \frac{d}{dx} (x e^{-x}) \\
	 \frac{d}{dx} (x e^{-x})	&= e^{-x} + x (-1)  e^{-x}  =  e^{-x} (1-x) \\
	L_1(x) 				&= e^x e^{-x} (1-x)  \\	 
	L_1(x) 				&=1-x \\	 
\ea

$n=2$
\ba
	L_2(x) 					&= \frac{e^x}{2!}  \frac{d^2}{dx^2} (x^2 e^{-x}) \\
	\frac{d}{dx} (x^2 e^{-x})		&= \frac{d}{dx}  (x (x e^{-x}))\\
							&= 1 (x e^{-x}) + x e^{-x} (1-x) \\
							&= x e^{-x} (2-x) \\
	\frac{d^2}{dx^2} (x^2 e^{-x})	&= \frac{d}{dx}(x e^{-x}) (2-x) + x e^{-x} \frac{d}{dx}  (2-x) \\
							&= e^{-x} (1-x) (2-x) + x e^{-x} (-1) \\
							&= e^{-x} [(1-x) (2-x) - x ] \\
							&=  e^{-x} (x^2 -4 x + 2) \\
	L_2(x) 					&= \frac{e^x}{2 \,1} e^{-x}  (x^2 -4 x + 2) \\
							&= 1 - 2 x + \frac{x^2}{2} \\							
\ea
\newpage
$n=3$
\ba
	L_3(x) 					&= \frac{e^x}{3!}  \frac{d^3}{dx^3} (x^3 e^{-x}) \\
	\frac{d}{dx} (x^3 e^{-x})		&= \frac{d}{dx}  (x (x^2 e^{-x}))\\
							&= 1 (x^2 e^{-x}) + x x e^{-x} (2-x) \\
							&= x^2  e^{-x} (3-x) \\
	\frac{d^2}{dx^2} (x^3 e^{-x})	&= \frac{d}{dx} \big( \frac{d}{dx}	(x^3 e^{-x}) \big )\\	
							&= \frac{d}{dx} \big ( x^2  e^{-x} (3-x) \big ) \\ 	
							&= \frac{d}{dx}  (x^2  e^{-x}) (3-x) +  ( x^2  e^{-x}) \frac{d}{dx} (3-x) \\ 	
							&=  x e^{-x} (2-x) (3-x) +  ( x^2  e^{-x}) (-1) \\	
							&=  x e^{-x} [ (2-x) (3-x) - x ] \\
							&=   x e^{-x} (x^2 -6 x + 6)\\	
	\frac{d^3}{dx^3} (x^3 e^{-x})	&= \frac{d}{dx} \big( \frac{d^2}{dx^2} (x^3 e^{-x}) \big )\\							
							&= \frac{d}{dx} \big( x e^{-x} (x^2 -6 x + 6)  \big )\\
							&=  \frac{d}{dx}  (x e^{-x}) (x^2 -6 x + 6) + x e^{-x}  \frac{d}{dx} (x^2 -6 x + 6) \\
							&= e^{-x} (1-x)  (x^2 -6 x + 6) + x e^{-x}  (2x-6) \\
							&= e^{-x}  (x^2-6x + 6 - x^3 + 6x^2 -6 x + 2 x^2 -6 x) \\
							&= e^{-x}  (-x^3 + 9 x^2 -18 x + 6)\\
	L_3(x)					&= \frac{e^x}{6}  e^{-x}  (-x^3 + 9 x^2 -18 x + 6)\\
							&= 1 -3x + \frac{3x^2}{2} - \frac{x^3}{6} \\
\ea

Show that the Laguerre polynomials are orthogonal on the positive axis ($0 \le x < \infty$) w.r.t. the weight function $e^{-x}$.
We want to show
\[
	\int_0^\infty L_n(x) L_k(x) e^{-x} dx = 0 ~ \text{ if } n \neq k
\]
From the expression of the Laguerre polynomial, $L_n(x) = \frac{e^x}{n!} \frac{d^n}{dx^n}(x^n e^{-x})$, 
a Laguerre polynomial $L_n(x)$ is a polynomial of degree $n$. Without loss of generality, let assume
in the previous equation that $k<n$. By multiplying two polynomials of degree $M$ and $N$, the result is a polynomial of degree at most $M + N$.
Therefore to prove orthogonality, suffices to prove the following equation:
\[
	\int_0^\infty  e^{-x} x^k L_n(x) dx = 0 ~ \text{ for all } k<n
\]
First we show that
\[
	\int_0^\infty   \frac{d^n}{dx^n} (x^m  e^{-x}) = 0 ~ \text{ for } n < m
\]

We are applying $n$th derivative rules:
\ba
	 \frac{d^n}{dx^n} (x^m  e^{-x}) 	&= 	\sum_{r=0}^n C_r^n \frac{d^r}{dx^r} x^n  \frac{d^{n-r}}{dx^{n-r}} e^{-x} \\
							&=	\sum_{r=0}^n \frac{n!}{r! (n-r)!} \frac{n!}{(n-r)!} x^{n-r} (-1)^{n-r}  e^{-x} \\
							&=	\sum_{r=0}^n  (-1)^{n-r}  \frac{(n!)^2}{r! ((n-r)!^2} x^{n-r} e^{-x} \\
\ea
reindexing with $l=n-r$, we obtain
\[
	 \frac{d^n}{dx^n} (x^m  e^{-x}) 	= 	\sum_{l=0}^m (-1)^l \frac{(n!)^2}{(l!)^2 (n-l)!} x^l e^{-x}
\]

Next
\[
	\int_0^\infty  \frac{d^n}{dx^n} (x^m  e^{-x}) 	= \sum_{l=0}^m (-1)^l \frac{(n!)^2}{(l!)^2 (n-l)!}   \int_0^\infty x^l e^{-x} dx
\]
Integration by parts l times gives:
\[
	\int_0^\infty x^l e^{-x} dx = (-1)^l l! \int_0^\infty e^{-x} dx =  (-1)^l l! [-e^{-x}]_0^\infty = 0
\]

Going back to our initial equation of interest, and applying integration by parts $k$ times:
\ba
	\int_0^\infty  e^{-x} x^k L_n(x) dx 	&= \int_0^\infty x^k  \frac{d^n}{dx^n}(x^n e^{-x}) dx \\
								&= [x^k \frac{d^{n-1}}{dx^{n-1}}(x^n e^{-x})]_0^\infty  - k \int_0^\infty x^{k-1}   \frac{d^{n-1}}{dx^{n-1}}(x^n e^{-x}) dx  \\
\ea
The first term is $0$ as it is a sum of terms $[x^{l+k} e^{-x}]_0^\infty = 0$ and after k iterations we are left with
\[
		\int_0^\infty  e^{-x} x^k L_n(x) dx  =(-1)^k k!  \int_0^\infty \frac{d^{n-k}}{dx^{n-k}}(x^n e^{-x}) dx  = 0
\]


\section*{Question 7}
\[
	x^2 \frac{d^2 y}{dx^2} + x \frac{dy}{dx} - y = x \, , y(e) = 0 \, , y'(e) = 2
\]

\be 
\item [a.]
This is Euler differential equation, and we make the change of variable $x=e^t$ or $t=\ln(x)$.
Then
\ba
	\frac{dy}{dx} &= \frac{dy}{dt} \frac{dt}{dx} = \frac{dy}{dt} \frac{d \ln x}{dx} =  \frac{dy}{dt} \frac{1}{x} = \frac{1}{x}   \frac{dy}{dt} \\ 
	x \frac{dy}{dx}  &= \frac{dy}{dt} \\
\ea
And since this is a Legendre ODE with $\alpha=1$ and $\beta=0$, we can use the expression for the second derivative 
$(\alpha x + \beta)^2 \frac{d^2y}{dx^2} = \alpha^2 \frac{d}{d t} [\frac{d}{dt} - 1] y$.
With $\alpha=1$ and $\beta=0$, we have: $ x^2 \frac{d^2y}{dx^2} =  \frac{d^2y}{dt^2} - \frac{dy}{dt}$. \\ \hfill \break

Substitute into the above equation yields:
\ba
	( \frac{d^2y}{dt^2} - \frac{dy}{dt}) + \frac{dy}{dt} - y &= e^t \\
	 \frac{d^2y}{dt^2} - y &= e^t \\
\ea

\item [b.]
The homogeneous equation is
\[
	 \frac{d^2y}{dt^2} - y = 0
\]

Assume a solution of the form $y(t) = A e^{\lambda t}$ gives the characteristic equation $\lambda^2 - 1 = 0$ which has for roots $\lambda = \pm1$ and gives
for solution $y(t) = c_1 e^t + c_2 e^{-t}$.

\item [c.]
The ODE to solve is:
\[
	 \frac{d^2y}{dt^2} - y = 0
\]
It is in standard form and it is defined at any point $t$, it is analytic, thus we take as solution $y(t) = \sum_{t=0}^\infty a_n t^n$.
So:
\ba
	y'(t)  &=  \sum_{t=0}^\infty n a_n t^{n-1} \\
	y''(t) &=  \sum_{t=0}^\infty n (n-1) a_n t^{n-2} \\
	\text{by reindexing}\\
	y''(t) &=  \sum_{t=-2}^\infty (n+2) (n+1) a_{n+2} t^n \\
	y''(t) &=  \sum_{t=0}^\infty (n+2) (n+1) a_{n+2} t^n \\
\ea
Substitute into the ODE gives:
\ba
	  \sum_{t=0}^\infty (n+2) (n+1) a_{n+2} t^n -  \sum_{t=0}^\infty a_n t^n &= 0\\
	  \sum_{t=0}^\infty [ (n+2) (n+1) a_{n+2} - a_n] t^n & = 0\\
\ea
or
\ba
a_{n+2} &= \frac{1}{(n+2) (n+1)} a_n \\
a_n &= \frac{1}{ n (n-1)} a_{n-2} \\
\ea
Take $a_0 = a_1 = 1$ and we generate the coefficients:
\be
	\item[.] $n = 2$ then $a_2 = \frac{1}{2 \cdot 1} a_0 = \frac{1}{2 \cdot 1} = \frac{1}{2!}$
	\item[.] $n = 3$ then $a_3 = \frac{1}{3 \cdot 2} a_1 = \frac{1}{3 \cdot 2} = \frac{1}{3!}$
	\item[.] $n = 4$ then $a_4 = \frac{1}{4 \cdot 3} a_2 = \frac{1}{4 \cdot 3 \cdot 2 \cdot 1} = \frac{1}{4!}$
	\item[$\vdots$]
	\item[.] $a_n = \frac{1}{ n (n-1)} a_{n-2} = \cdots = \frac{1}{n!}$
\ee
The first solution we obtain is: $y_1(t) =  \sum_{t=0}^\infty a_n t^n =  \sum_{t=0}^\infty \frac{t^n}{n!} = e^t$.
Secondly, if we set $a_0=1$ and choose $a_1=-1$, then we obtain a second independent solution:
\be
	\item[.] $n = 2$ then $a_2 = \frac{1}{2 \cdot 1} a_0 = \frac{1}{2 \cdot 1} = \frac{1}{2!}$
	\item[.] $n = 3$ then $a_3 = \frac{1}{3 \cdot 2} a_1 = - \frac{1}{3 \cdot 2} = \frac{-1}{3!}$
	\item[.] $n = 4$ then $a_4 = \frac{1}{4 \cdot 3} a_2 = \frac{1}{4 \cdot 3 \cdot 2 \cdot 1} = \frac{1}{4!}$
	\item[.] $n = 5$ then $a_5 = \frac{1}{5 \cdot 4} a_3 = \frac{-1}{5 \cdot 4 \cdot 3 \cdot 2 \cdot 1} = \frac{-1}{5!}$
	\item[$\vdots$]
	\item[.] $a_n = \frac{1}{ n (n-1)} a_{n-2} = \cdots = \frac{(-1)^n}{n!}$
\ee
We have the second solution: $y_2(t) =  \sum_{t=0}^\infty a_n t^n =  \sum_{t=0}^\infty \frac{(-t)^n}{n!}$, 
recognizing the last series as $e^{-t}$, we can write the general solution of the homogeneous equation as 
\[
	y_H(t) = c_1 e^t + c_2 e^{-t}
\]
which is the solution we found in question b.

\item[d.]
The differential equation to solve is
\[
	 \frac{d^2y}{dt^2} - y = e^t
\]
Next we use the variation of parameters method, we are looking for a solution $y_p(t) = k_1(t) e^t + k_2(t) e^{-t}$.
We solve for derivatives of k's a system of two equations:
\begin{align*}
&
	\begin{cases}
		k'_1 e^t + k'_2 e^{-t} & = 0 \\
		k'_1 e^t - k'_2 e^{-t} & = e^t \\
	\end{cases}
\end{align*}
Multiplying through by $e^t$ gives:
\begin{align*}
&
	\begin{cases}
		k'_1 e^{2t} + k'_2 & = 0 \\
		k'_1 e^{2t} - k'_2  & = e^{2t} \\
	\end{cases}
\end{align*}

Adding first equation to second yields $2 k'_1 e^{2t} = e^{2t}$ or $k'_1 = \frac{1}{2}$ and $k_1 = \frac{t}{2}$.
Substitute
\ba
	k'_2 &= - k'_1 e^{2t} \\
		&= - \frac{1}{2} e^{2t}  \\
	\text{integrating} \\
	k_2 &= - \frac{e^{2t}}{4} \\
\ea
Therefore:
\ba
	y_p(t) 	&= k_1(t) e^t + k_2(t) e^{-t} \\
			&=  \frac{t}{2} e^t - \frac{e^{2t}}{4} e^{-t} \\
			&=  \frac{t}{2} e^t - \frac{e^t}{4} \\
			&= \frac{e^t}{2} (t - \frac{1}{2}) \\
\ea

\item [e.] The general solution is: $y(t) =  y_H(t) + y_p(t) = c_1 e^t + c_2 e^{-t} + \frac{e^t}{2} (t - \frac{1}{2})$, simplifying the constants,
we can rewrite the general solution as $y(t) =  c_1 e^t + c_2 e^{-t} + \frac{t}{2} e^t$.
Plugging back $x=e^t$ or $t=\ln(x)$ gives 
\[
	y(x) = c_1 x + \frac{c_2}{x} + \frac{x \ln x}{ 2}
\]

\item [f.]
The total solution is
\ba
	y(x) 	&= c_1 x + \frac{c_2}{x} + \frac{x \ln x}{2} \\
	y'(x) 	&= c_1 - \frac{c_2}{x^2} + \frac{1}{2} (1 + \ln x) \\
\ea
And the initial conditions are $y(e) = 0 \, , y'(e) = 2$, plugging back these into the previous equations gives 
\ba
	\begin{cases}
	y(e) = c_1 e + \frac{c_2}{e} + \frac{e \ln e} {2} = 0 \\
	y'(e) = c_1 - \frac{c_2}{e^2} + \frac{1} {2} (1 +  \ln e)= 2 \\
	\end{cases} \\	
\ea
\ba
	\Rightarrow 
	\begin{cases}
	c_1 e + c_2 e^{-1}  = - \frac{e}{2} \\
	c_1 - c_2 e^{-2} = 1 \\
	\end{cases} \\	
\ea
\ba
	\Rightarrow 
	\begin{cases}
	c_1 e^2 + c_2  = - \frac{e^2}{2} \\
	c_1 e^2 - c_2  = e^2 \\
	\end{cases} \\	
\ea

Adding the first equation to the second equation leads to $2 e^2 c_1 = e^2 - \frac{e^2}{2} =  \frac{e^2}{2}$,  $c_1=\frac{1}{4}$,$c_2 = e^2 (c_1-1) = \frac{3}{4} e^2$. 
Reporting these constants into the expression of the total solution gives:
\[
	y(x) = \frac{1}{4} x -  \frac{3}{4} e^2 \frac{1}{x} + \frac{x \ln x}{2} 
\]
\[
	y(x) = \frac{x^2 + 2 x^2 \ln(x) - 3 e^2} { 4 x} 
\]

\ee
\end{document}
