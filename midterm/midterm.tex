\documentclass[12pt,twoside]{article}
\usepackage[dvipsnames]{xcolor}
\usepackage{tikz,graphicx,amsmath,amsfonts,amscd,amssymb,mathrsfs, bm,cite,epsfig,epsf,url}
\usepackage[hang,flushmargin]{footmisc}
\usepackage[colorlinks=true,urlcolor=blue,citecolor=blue]{hyperref}
\usepackage{amsthm,multirow,wasysym,appendix}
\usepackage{array,subcaption} 
% \usepackage[small,bf]{caption}
\usepackage{bbm}
\usepackage{pgfplots}
\usetikzlibrary{spy}
\usepgfplotslibrary{external}
\usepgfplotslibrary{fillbetween}
\usetikzlibrary{arrows,automata}
\usepackage{thmtools}
\usepackage{blkarray} 
\usepackage{textcomp}
\usepackage[left=0.8in,right=1.0in,top=1.0in,bottom=1.0in]{geometry}


\usepackage{times}
\usepackage{amsfonts}
\usepackage{amsmath}
\usepackage{latexsym}
\usepackage{color}
\usepackage{graphics}
\usepackage{enumerate}
\usepackage{amstext}
\usepackage{blkarray}
\usepackage{url}
\usepackage{epsfig}
\usepackage{bm}
\usepackage{hyperref}
\hypersetup{
    colorlinks=true,
    linkcolor=blue,
    filecolor=magenta,      
    urlcolor=blue,
}
\usepackage{textcomp}
\usepackage[left=0.8in,right=1.0in,top=1.0in,bottom=1.0in]{geometry}
\usepackage{mathtools}
\usepackage{minted}



\input{macros}

\begin{document}

\noindent Professor Rio\\
EN.585.615.81.SP21 Mathematical Methods\\
Mid-term Exam\\
Johns Hopkins University\\
Student: Yves Greatti\\\

\section*{Question 7}
\[
	x^2 \frac{d^2 y}{dx^2} + x \frac{dy}{d} - y = x \, , y(e) = 0 \, , y'(e) = 2
\]

\be 
\item [a.]
This is Euler differential equation, and we make the change of variable $x=e^t$ or $t=\ln(x)$.
Then
\ba
	\frac{dy}{dx} &= \frac{dy}{dt} \frac{dt}{dx} = \frac{dy}{dt} \frac{d \ln x}{dx} =  \frac{dy}{dt} \frac{1}{x} = \frac{1}{x}   \frac{dy}{dt} \\ 
	x \frac{dy}{dx}  &= \frac{dy}{dt} \\
\ea
And since this is a Legendre ODE with $\alpha=1$ and $\beta=0$, we can use the expression for the second derivative 
$(\alpha x + \beta)^2 \frac{d^2y}{dx^2} = \alpha^2 \frac{d}{d t} [\frac{d}{dt} - 1] y$.
With $\alpha=1$ and $\beta=0$, we have: $ \frac{d^2y}{dx^2} =  \frac{d^2y}{ t^2} - \frac{dy}{dt}$. \\ \hfill \break

Substitute into the above equation yields:
\ba
	( \frac{d^2y}{dt^2} - \frac{dy}{dt}) + \frac{dy}{dt} - y &= e^t \\
	 \frac{d^2y}{dt^2} - y &= e^t \\
\ea

\item [b.]
The homogeneous equation is
\[
	 \frac{d^2y}{dt^2} - y = 0
\]

Assume a solution of the form $y(t) = A e^{\lambda t}$ gives the characteristic equation $\lambda^2 - 1 = 0$ which has for roots $\lambda = \pm1$ and gives
for solution $y(t) = c_1 e^t + c_2 e^{-t}$.

\item [c.]
The ODE to solve is:
\[
	 \frac{d^2y}{dt^2} - y = 0
\]
It is in standard form and it is defined at any point $t$, it is analytic, thus we take as solution $y(t) = \sum_{t=0}^\infty a_n t^n$.
So:
\ba
	y'(t)  &=  \sum_{t=0}^\infty n a_n t^{n-1} \\
	y''(t) &=  \sum_{t=0}^\infty n (n-1) a_n t^{n-2} \\
	\text{by reindexing}\\
	y''(t) &=  \sum_{t=-2}^\infty (n+2) (n+1) a_{n+2} t^n \\
	y''(t) &=  \sum_{t=0}^\infty (n+2) (n+1) a_{n+2} t^n \\
\ea
Substitute into the ODE gives:
\ba
	  \sum_{t=0}^\infty (n+2) (n+1) a_{n+2} t^n -  \sum_{t=0}^\infty a_n t^n &= 0\\
	  \sum_{t=0}^\infty [ (n+2) (n+1) a_{n+2} - a_n] t^n & = 0\\
\ea
or
\ba
a_{n+2} &= \frac{1}{(n+2) (n+1)} a_n \\
a_n &= \frac{1}{ n (n-1)} a_{n-2} \\
\ea
Take $a_0 = a_1 = 1$ and we generate the coefficients:
\be
	\item[.] $n = 2$ then $a_2 = \frac{1}{2 \cdot 1} a_0 = \frac{1}{2 \cdot 1} = \frac{1}{2!}$
	\item[.] $n = 3$ then $a_3 = \frac{1}{3 \cdot 2} a_1 = \frac{1}{3 \cdot 2} = \frac{1}{3!}$
	\item[.] $n = 4$ then $a_4 = \frac{1}{4 \cdot 3} a_2 = \frac{1}{4 \cdot 3 \cdot 2 \cdot 1} = \frac{1}{4!}$
	\item[$\vdots$]
	\item[.] $a_n = \frac{1}{ n (n-1)} a_{n-2} = \cdots = \frac{1}{n!}$
\ee
The first solution we obtain is: $y_1(t) =  \sum_{t=0}^\infty a_n t^n =  \sum_{t=0}^\infty \frac{t^n}{n!} = e^t$.
Secondly, if we set $a_0=1$ and choose $a_1=-1$, then we obtain a second independent solution:
\be
	\item[.] $n = 2$ then $a_2 = \frac{1}{2 \cdot 1} a_0 = \frac{1}{2 \cdot 1} = \frac{1}{2!}$
	\item[.] $n = 3$ then $a_3 = \frac{1}{3 \cdot 2} a_1 = - \frac{1}{3 \cdot 2} = \frac{-1}{3!}$
	\item[.] $n = 4$ then $a_4 = \frac{1}{4 \cdot 3} a_2 = \frac{1}{4 \cdot 3 \cdot 2 \cdot 1} = \frac{1}{4!}$
	\item[.] $n = 5$ then $a_5 = \frac{1}{5 \cdot 4} a_3 = \frac{-1}{5 \cdot 4 \cdot 3 \cdot 2 \cdot 1} = \frac{-1}{5!}$
	\item[$\vdots$]
	\item[.] $a_n = \frac{1}{ n (n-1)} a_{n-2} = \cdots = \frac{(-1)^n}{n!}$
\ee
We have the second solution: $y_2(t) =  \sum_{t=0}^\infty a_n t^n =  \sum_{t=0}^\infty \frac{(-t)^n}{n!}$, 
recognizing the last series as $e^{-t}$, we can write the general solution of the homogeneous equation as 
\[
	y_H(t) = c_1 e^t + c_2 e^{-t}
\]
which is the solution we found in question b.

\item[d.]
The differential equation to solve is
\[
	 \frac{d^2y}{dt^2} - y = e^t
\]
Next we use the variation of parameters method, we are looking for a solution $y_p(t) = k_1(t) e^t + k_2(t) e^{-t}$.
We solve for derivatives of k's a system of two equations:
\begin{align*}
&
	\begin{cases}
		k'_1 e^t + k'_2 e^{-t} & = 0 \\
		k'_1 e^t - k'_2 e^{-t} & = e^t \\
	\end{cases}
\end{align*}
Multiplying through by $e^t$ gives:
\begin{align*}
&
	\begin{cases}
		k'_1 e^{2t} + k'_2 & = 0 \\
		k'_1 e^{2t} - k'_2  & = e^{2t} \\
	\end{cases}
\end{align*}

Adding first equation to second yields $2 k'_1 e^{2t} = e^{2t}$ or $k'_1 = \frac{1}{2}$ and $k_1 = \frac{t}{2}$.
Substitute
\ba
	k'_2 &= - k'_1 e^{2t} \\
		&= - \frac{1}{2} e^{2t}  \\
	\text{integrating} \\
	k_2 &= - \frac{e^{2t}}{4} \\
\ea
Therefore:
\ba
	y_p(t) 	&= k_1(t) e^t + k_2(t) e^{-t} \\
			&=  \frac{t}{2} e^t - \frac{e^{2t}}{4} e^{-t} \\
			&=  \frac{t}{2} e^t - \frac{e^t}{4} \\
			&= \frac{e^t}{2} (t - \frac{1}{2}) \\
\ea

\item [e.] The general solution is: $y(t) =  y_H(t) + y_p(t) = c_1 e^t + c_2 e^{-t} + \frac{e^t}{2} (t - \frac{1}{2})$, simplifying the constants,
we can rewrite the general solution as $y(t) =  c_1 e^t + c_2 e^{-t} + \frac{t}{2} e^t$.
Plugging back $x=e^t$ or $t=\ln(x)$ gives 
\[
	y(x) = c_1 x + \frac{c_2}{x} + \frac{x \ln x}{ 2}
\]

\item [f.]
The total solution is
\ba
	y(x) 	&= c_1 x + \frac{c_2}{x} + \frac{x \ln x}{2} \\
	y'(x) 	&= c_1 x - \frac{c_2}{x^2} + \frac{1}{2} (1 + \ln x) \\
\ea
And the initial conditions are $y(e) = 0 \, , y'(e) = 2$, plugging back these into the previous equations gives 
\ba
	\begin{cases}
	y(e) = c_1 e + \frac{c_2}{e} + \frac{e \ln e} {2} = 0 \\
	y'(e) = c_1 - \frac{c_2}{e^2} + \frac{1} {2} (1 +  \ln e)= 2 \\
	\end{cases} \\	
\ea
\ba
	\Rightarrow 
	\begin{cases}
	c_1 e + c_2 e^{-1}  = - \frac{e}{2} \\
	c_1 - c_2 e^{-2} = 1 \\
	\end{cases} \\	
\ea
\ba
	\Rightarrow 
	\begin{cases}
	c_1 e^2 + c_2  = - \frac{e^2}{2} \\
	c_1 - c_2 e^{-2} = 1 \\
	\end{cases} \\	
\ea
Adding equation (1) to equation (2) leads to $2 c_1 = e^2 - \frac{e^2}{2} =  \frac{e^2}{2}$,  $c_1=\frac{1}{4}$,$c_2 = e^2 (c_1-1) = \frac{3}{4} e^2$. 
Reporting these constants into the expression of the total solution gives:
\[
	y(x) = \frac{1}{4} x -  \frac{3}{4} e^2 \frac{1}{x} + \frac{x \ln x}{2} 
\]
\[
	y(x) = \frac{x^2 + 2 x^2 \ln(x) - 3 e^2} { 4 x} 
\]

\ee
\end{document}
