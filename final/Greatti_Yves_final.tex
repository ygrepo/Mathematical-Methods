\documentclass[12pt,twoside]{article}
\usepackage[dvipsnames]{xcolor}
\usepackage{tikz,graphicx,amsmath,amsfonts,amscd,amssymb,mathrsfs, bm,cite,epsfig,epsf,url}
\usepackage[hang,flushmargin]{footmisc}
\usepackage[colorlinks=true,urlcolor=blue,citecolor=blue]{hyperref}
\usepackage{amsthm,multirow,wasysym,appendix}
\usepackage{array,subcaption} 
% \usepackage[small,bf]{caption}
\usepackage{bbm}
\usepackage{pgfplots}
\usetikzlibrary{spy}
\usepgfplotslibrary{external}
\usepgfplotslibrary{fillbetween}
\usetikzlibrary{arrows,automata}
\usepackage{thmtools}
\usepackage{blkarray} 
\usepackage{textcomp}
\usepackage[left=0.8in,right=1.0in,top=1.0in,bottom=1.0in]{geometry}


\usepackage{times}
\usepackage{amsfonts}
\usepackage{amsmath}
\usepackage{latexsym}
\usepackage{color}
\usepackage{graphics}
\usepackage{enumerate}
\usepackage{amstext}
\usepackage{blkarray}
\usepackage{url}
\usepackage{epsfig}
\usepackage{bm}
\usepackage{hyperref}
\hypersetup{
    colorlinks=true,
    linkcolor=blue,
    filecolor=magenta,      
    urlcolor=blue,
}
\usepackage{textcomp}
\usepackage[left=0.8in,right=1.0in,top=1.0in,bottom=1.0in]{geometry}
\usepackage{mathtools}
%\usepackage{minted}



%% Probability operators and functions
%
% \def \P{\mathrm{P}}
\def \P{\mathrm{P}}
\def \E{\mathrm{E}}
\def \Var{\mathrm{Var}}
\let\var\Var
\def \Cov {\mathrm{Cov}} \let\cov\Cov
\def \MSE {\mathrm{MSE}} \let\mse\MSE
\def \sgn {\mathrm{sgn}}
\def \R {\mathbb{R}}
\def \C {\mathbb{C}}
\def \N {\mathbb{N}}
\def \Z {\mathbb{Z}}
\def \cV {\mathcal{V}}
\def \cS {\mathcal{S}}

\newcommand{\RR}{\ensuremath{\mathbb{R}}}

\DeclareMathOperator*{\argmin}{arg\,min}
\DeclareMathOperator*{\argmax}{arg\,max}
\newcommand{\red}[1]{\textcolor{red}{#1}}
\newcommand{\blue}[1]{\textcolor{blue}{#1}}
\newcommand{\green}[1]{\textcolor{ForestGreen}{ #1}}
\newcommand{\fuchsia}[1]{\textcolor{RoyalPurple}{ #1}}



%
%% Probability distributions
%
%\def \Bern    {\mathrm{Bern}}
%\def \Binom   {\mathrm{Binom}}
%\def \Exp     {\mathrm{Exp}}
%\def \Geom    {\mathrm{Geom}}
% \def \Norm    {\mathcal{N}}
%\def \Poisson {\mathrm{Poisson}}
%\def \Unif    {\mathrm {U}}
%
\DeclareMathOperator{\Norm}{\mathcal{N}}

\newcommand{\bdb}[1]{\textcolor{red}{#1}}

\newcommand{\ml}[1]{\mathcal{ #1 } }
\newcommand{\wh}[1]{\widehat{ #1 } }
\newcommand{\wt}[1]{\widetilde{ #1 } }
\newcommand{\conj}[1]{\overline{ #1 } }
\newcommand{\rnd}[1]{\tilde{ #1 } }
\newcommand{\rv}[1]{ \rnd{ #1}  }
\newcommand{\rx}{\rnd{ x}  }
\newcommand{\ry}{\rnd{ y}  }
\newcommand{\rz}{\rnd{ z}  }
\newcommand{\ra}{\rnd{ a}  }
\newcommand{\rb}{\rnd{ b}  }
\newcommand{\rpc}{\widetilde{ pc}  }
\newcommand{\rndvec}[1]{\vec{\rnd{#1}}}

\def \cnd {\, | \,}
\def \Id { I }
\def \J {\mathbf{1}\mathbf{1}^T}

\newcommand{\op}[1]{\operatorname{#1}}
\newcommand{\setdef}[2]{ := \keys{ #1 \; | \; #2 } }
%\newcommand{\set}[2]{ \keys{ #1 \; | \; #2 } }
\newcommand{\sign}[1]{\op{sign}\left( #1 \right) }
\newcommand{\trace}[1]{\op{tr}\left( #1 \right) }
\newcommand{\tr}[1]{\op{tr}\left( #1 \right) }
\newcommand{\inv}[1]{\left( #1 \right)^{-1} }
%\newcommand{\abs}[1]{\left| #1 \right|}
\newcommand{\sabs}[1]{| #1 |}
\newcommand{\keys}[1]{\left\{ #1 \right\}}
\newcommand{\sqbr}[1]{\left[ #1 \right]}
\newcommand{\sbrac}[1]{ ( #1 ) }
\newcommand{\brac}[1]{\left( #1 \right) }
\newcommand{\bbrac}[1]{\big( #1 \big) }
\newcommand{\Bbrac}[1]{\Big( #1 \Big)}
\newcommand{\BBbrac}[1]{\BIG( #1 \Big)}
\newcommand{\MAT}[1]{\begin{bmatrix} #1 \end{bmatrix}}
\newcommand{\sMAT}[1]{\left(\begin{smallmatrix} #1 \end{smallmatrix}\right)}
\newcommand{\sMATn}[1]{\begin{smallmatrix} #1 \end{smallmatrix}}
\newcommand{\PROD}[2]{\left \langle #1, #2\right \rangle}
\newcommand{\PRODs}[2]{\langle #1, #2 \rangle}
\newcommand{\der}[2]{\frac{\text{d}#2}{\text{d}#1}}
\newcommand{\pder}[2]{\frac{\partial#2}{\partial#1}}
\newcommand{\derTwo}[2]{\frac{\text{d}^2#2}{\text{d}#1^2}}
\newcommand{\ceil}[1]{\lceil #1 \rceil}
\newcommand{\Imag}[1]{\op{Im}\brac{ #1 }}
\newcommand{\Real}[1]{\op{Re}\brac{ #1 }}
%\newcommand{\norm}[1]{\left|\left| #1 \right|\right| }
\newcommand{\norms}[1]{ \| #1 \|  }
\newcommand{\normProd}[1]{\left|\left| #1 \right|\right| _{\PROD{\cdot}{\cdot}} }
\newcommand{\normTwo}[1]{\left|\left| #1 \right|\right| _{2} }
\newcommand{\normTwos}[1]{ \| #1  \| _{2} }
\newcommand{\normZero}[1]{\left|\left| #1 \right|\right| _{0} }
\newcommand{\normTV}[1]{\left|\left| #1 \right|\right|  _{ \op{TV}  } }% _{\op{c} \ell_1} }
\newcommand{\normOne}[1]{\left|\left| #1 \right|\right| _{1} }
\newcommand{\normOnes}[1]{\| #1 \| _{1} }
\newcommand{\normOneTwo}[1]{\left|\left| #1 \right|\right| _{1,2} }
\newcommand{\normF}[1]{\left|\left| #1 \right|\right| _{\op{F}} }
\newcommand{\normLTwo}[1]{\left|\left| #1 \right|\right| _{\ml{L}_2} }
\newcommand{\normNuc}[1]{\left|\left| #1 \right|\right| _{\ast} }
\newcommand{\normOp}[1]{\left|\left| #1 \right|\right|  }
\newcommand{\normInf}[1]{\left|\left| #1 \right|\right| _{\infty}  }
\newcommand{\proj}[1]{\mathcal{P}_{#1} \, }
\newcommand{\diff}[1]{ \, \text{d}#1 }
\newcommand{\vc}[1]{\boldsymbol{\vec{#1}}}
\newcommand{\rc}[1]{\boldsymbol{#1}}
\newcommand{\vx}{\vec{x}}
\newcommand{\vy}{\vec{y}}
\newcommand{\vz}{\vec{z}}
\newcommand{\vu}{\vec{u}}
\newcommand{\vv}{\vec{v}}
\newcommand{\vb}{\vec{\beta}}
\newcommand{\va}{\vec{\alpha}}
\newcommand{\vaa}{\vec{a}}
\newcommand{\vbb}{\vec{b}}
\newcommand{\vg}{\vec{g}}
\newcommand{\vw}{\vec{w}}
\newcommand{\vh}{\vec{h}}
\newcommand{\vbeta}{\vec{\beta}}
\newcommand{\valpha}{\vec{\alpha}}
\newcommand{\vgamma}{\vec{\gamma}}
\newcommand{\veta}{\vec{\eta}}
\newcommand{\vnu}{\vec{\nu}}
\newcommand{\rw}{\rnd{w}}
\newcommand{\rvnu}{\vc{\nu}}
\newcommand{\rvv}{\rndvec{v}}
\newcommand{\rvw}{\rndvec{w}}
\newcommand{\rvx}{\rndvec{x}}
\newcommand{\rvy}{\rndvec{y}}
\newcommand{\rvz}{\rndvec{z}}
\newcommand{\rvX}{\rndvec{X}}


\newtheorem{theorem}{Theorem}[section]
% \declaretheorem[style=plain,qed=$\square$]{theorem}
\newtheorem{corollary}[theorem]{Corollary}
\newtheorem{definition}[theorem]{Definition}
\newtheorem{lemma}[theorem]{Lemma}
\newtheorem{remark}[theorem]{Remark}
\newtheorem{algorithm}[theorem]{Algorithm}

% \theoremstyle{definition}
%\newtheorem{example}[proof]{Example}
\declaretheorem[style=definition,qed=$\triangle$,sibling=definition]{example}
\declaretheorem[style=definition,qed=$\bigcirc$,sibling=definition]{application}

%
%% Typographic tweaks and miscellaneous
%\newcommand{\sfrac}[2]{\mbox{\small$\displaystyle\frac{#1}{#2}$}}
%\newcommand{\suchthat}{\kern0.1em{:}\kern0.3em}
%\newcommand{\qqquad}{\kern3em}
%\newcommand{\cond}{\,|\,}
%\def\Matlab{\textsc{Matlab}}
%\newcommand{\displayskip}[1]{\abovedisplayskip #1\belowdisplayskip #1}
%\newcommand{\term}[1]{\emph{#1}}
%\renewcommand{\implies}{\;\Rightarrow\;}

% My macros

\def\Kset{\mathbb{K}}
\def\Nset{\mathbb{N}}
\def\Qset{\mathbb{Q}}
\def\Rset{\mathbb{R}}
\def\Sset{\mathbb{S}}
\def\Zset{\mathbb{Z}}
\def\squareforqed{\hbox{\rlap{$\sqcap$}$\sqcup$}}
\def\qed{\ifmmode\squareforqed\else{\unskip\nobreak\hfil
\penalty50\hskip1em\null\nobreak\hfil\squareforqed
\parfillskip=0pt\finalhyphendemerits=0\endgraf}\fi}

%\DeclareMathOperator*{\E}{\rm E}
%\DeclareMathOperator*{\argmax}{\rm argmax}
%\DeclareMathOperator*{\argmin}{\rm argmin}
%\DeclareMathOperator{\sgn}{sign}
\DeclareMathOperator{\supp}{supp}
\DeclareMathOperator{\last}{last}
%\DeclareMathOperator{\sign}{\sgn}
\DeclareMathOperator{\diag}{diag}
\providecommand{\abs}[1]{\lvert#1\rvert}
\providecommand{\norm}[1]{\lVert#1\rVert}
\def\vcdim{\textnormal{VCdim}}
\DeclareMathOperator*{\B}{\textbf{B}}

%\DeclarePairedDelimiter\ceil{\lceil}{\rceil}
%\DeclarePairedDelimiter\floor{\lfloor}{\rfloor}

\newcommand{\cX}{{\mathcal X}}
\newcommand{\cY}{{\mathcal Y}}
\newcommand{\cA}{{\mathcal A}}
\newcommand{\ignore}[1]{}
\newcommand{\ba}{\[\begin{aligned}}
\newcommand{\ea}{\end{aligned}\]}
\newcommand{\bi}{\begin{itemize}}
\newcommand{\ei}{\end{itemize}}
\newcommand{\be}{\begin{enumerate}}
\newcommand{\ee}{\end{enumerate}}
\newcommand{\bd}{\begin{description}}
\newcommand{\ed}{\end{description}}
\newcommand{\h}{\widehat}
\newcommand{\e}{\epsilon}
\newcommand{\mat}[1]{{\mathbf #1}}
%\newcommand{\R}{\mat{R}}
\newcommand{\0}{\mat{0}}
\newcommand{\M}{\mat{M}}

\newcommand{\D}{\mat{D}}
\renewcommand{\r}{\mat{r}}
\newcommand{\x}{\mat{x}}
\renewcommand{\u}{\mat{u}}
\renewcommand{\v}{\mat{v}}
\newcommand{\w}{\mat{w}}
\renewcommand{\H}{\text{0}}
\newcommand{\T}{\text{1}}
%\newcommand{\set}[1]{\{#1\}}
\newcommand{\xxi}{{\boldsymbol \xi}}
\newcommand{\ssigma}{{\boldsymbol \sigma}}
\newcommand{\Alpha}{{\boldsymbol \alpha}}
\newcommand{\tts}{\tt \small}
\newcommand{\hint}{\emph{hint}}
\newcommand{\matr}[1]{\bm{#1}}     % ISO complying version
\newcommand{\vect}[1]{\bm{#1}} % vectors

%\newcommand{\Var}{\mathrm{Var}}
%\newcommand{\Cov}{\mathrm{Cov}}

% New commands
\newcommand{\SP}{\mathbf{S}_{+}^n}
\newcommand{\Proj}{\mathcal{P}_{\mathcal{S}}}
%\DeclarePairedDelimiterX{\inp}[2]{\langle}{\rangle}{#1, #2}



\begin{document}

\noindent Professor Rio\\
EN.585.615.81.SP21 Mathematical Methods\\
Final Exam\\
Johns Hopkins University\\
Student: Yves Greatti\\\

\section*{Question 1}
\be
\item [a.]
$f(x) = x$ is odd on $[-\pi,\pi]$ therefore its Fourier coefficients $a_n$ are $0$ and we need to find its $b_n$ coefficients:
 \ba
 	b_n	&= \frac{2}{2\pi} \int_{-\pi}^{\pi} f(x) \sin(\frac{2 \pi n x}{2 \pi}) dx \\
		&= \frac{4}{2\pi} \int_{0}^{\pi} x \sin(\frac{2 \pi n x}{2 \pi}) dx \\
		&= \frac{2}{\pi} \int_{0}^{\pi} x \sin(n x) dx \\
 \ea
 Using integration by parts:
 \ba
 	 \int_{0}^{\pi} x \sin(n x) dx &= [x (-\frac{\cos(nx)}{n}) ]_0^\pi + \int_{0}^{\pi} 1 \cdot \frac{\cos(nx)}{n} dx \\
	 					&=( -\frac{\pi}{n}) \cos(n \pi) + \frac{1}{n} [\sin(nx)]_0^\pi \\
						&= \frac{(-1)^{n+1} \pi}{n} \\
 \ea
 Thus $b_n = \frac{2}{\pi}   \frac{(-1)^{n+1} \pi}{n} = \frac{(-1)^{n+1} 2}{n}$ and the Fourier series of $x$, on $[-\pi,\pi]$, is:
 \[
 	x = \sum_{n=1}^\infty b_n \sin(nx) = 2  \sum_{n=1}^\infty \frac{(-1)^{n+1}  \sin(nx)}{n}
 \]
 
\item [b.]
If we integrate terms by terms the previous expression, the Fourier series of $x$ over  $[-\pi,\pi]$, we have:
\ba
	\frac{x^2}{2}	&= 2 \sum_{n=1}^\infty \frac{(-1)^{n+1}}{n} (-\frac{\cos(nx)}{n}) + c ~ ~ c\text{constant of integration} \\
			x^2	&= 4 \sum_{n=1}^\infty \frac{(-1)^n}{n^2} \cos(nx) + c \text{ with } 2 c \rightarrow c \\
				&= c + 4 \sum_{n=1}^\infty \frac{(-1)^n}{n^2} \cos(nx)
\ea

\item [c.]
$f(x)=x^2$ is an even function, by Fourier Series for even function over symmetric range, we have:
\[
	x^2 = \frac{a_0}{2} + \sum_{n=1}^\infty a_n \cos{(\frac{2 \pi n x}{2 \pi})} = \frac{a_0}{2}  + \sum_{n=1}^\infty a_n \cos(nx) ~ (1)
\]
where
\ba
	a_0	&= \frac{4}{2\pi} \int_0^\pi x^2 dx \\
		&=  \frac{2}{\pi} [\frac{x^3}{3}]_0^\pi \\
		&= \frac{2}{3} \pi^2 \\
		\\
	a_n 					&= \frac{4}{2\pi} \int_0^\pi x^2 \cos{(\frac{2 \pi n x}{2 \pi})} dx = \frac{2}{\pi} \int_0^\pi x^2 \cos(nx) dx \\
	 \int_0^\pi x^2 \cos(nx) dx 	&= [x^2 ~ \frac{\sin(nx)}{n}]_0^\pi -\frac{2}{n} \int_0^\pi x \sin(nx) dx \\
	 					&= 0 -\frac{2}{n}  \frac{(-1)^{n+1} \pi}{n} \\
	a_n					&=  \frac{2}{\pi} \frac{(-1)^n 2 ~ \pi}{n^2} \\
						&= (-1)^n \frac{4}{n^2} \\
\ea
Substituting for $a_n$ in (1):
\[
	x^2 = \frac{\pi^2}{3} + 4 \sum_{n=1}^\infty \frac{(-1)^n}{n^2} \cos(nx)
\]

\item [d.]
Fourier series of $x^2$ using integration terms by terms or calculating directly match, as required, by taking $c=\frac{\pi^2}{3}$ since $x$ is a piecewise smooth function on the specified range.

\ee

\section*{Question 2}
Consider the differential equation:
\[
	z \frac{d^2y}{dy^2} + y = 0
\]

\be
\item [a.]
We put the equation in standard form:
\[
	\frac{d^2y}{dy^2} + \frac{1}{z} y = 0
\]

$z~p(z)=0$ and $z^2 q(z) = z$ therefore $0$ is a regular singular point.

\item [b.]
Take $y= z^\sigma \sum_{n=0}^\infty a_n z^n$ and the usual derivatives in the D.E. gives by substitution
\ba
	z \sum_{n=0}^\infty  (n+\sigma) (n+\sigma - 1) a_n z^{n+\sigma-2} +  \sum_{n=0}^\infty a_n z^{n+\sigma}	&= 0 \\
	\sum_{n=0}^\infty  (n+\sigma) (n+\sigma - 1) a_n z^{n+\sigma-1} +  \sum_{n=0}^\infty a_n z^{n+\sigma}	&= 0 ~ (1) \\
\ea
Take the term with the lowest power of $z$, which is the first sum with $n=0$, then since each power of $z$ term must be equal to $0$, we have
\[
	\sigma (\sigma-1) a_0 z^{\sigma-1} = 0
\]
Since $a_0 \neq 0$ and $ z^{\sigma-1} \neq 0$, therefore $\sigma =0,1$. 

\item [c.]
We go back to equation $(1)$ and take $\sigma=1$ yields
\[
	\sum_{n=0}^\infty  n (n+1)  a_n z^n +  \sum_{n=0}^\infty a_n z^{n+1} = 0 
\]
Then reindex the second sum to get same power of $z$ in both sums:
\[
	\sum_{n=0}^\infty  n (n+1)  a_n z^n +  \sum_{n=1}^\infty a_{n-1} z^n = 0 
\]
Note, in first term $n = 0$ does not contribute so we can start index at $n = 1$ in the first sum, and combine both sums

\[
	\sum_{n=1}^\infty  [n (n+1)  a_n +  a_{n-1}] z^n = 0
\]
Since every power of $z$ term must be $0$ and $z^n \neq 0$, gives:
\[
	a_n = - \frac{1}{(n+1)n} a_{n-1}
\]
Taking $a_0=1$, now 
\ba
	n =1 ~ & a_1 = - \frac{1}{2 ~ 1} a_0 = - \frac{1}{2 ~ 1} = \frac{(-1)^1}{2 ~ 1} \\
	n =2 ~ & a_2 = - \frac{1}{3 ~ 2} a_1 = \frac{1}{3 ~ 2 ~ 2 ~ 1} = \frac{(-1)^2}{(3 ~ 2 ~ 1) ~ (2 ~ 1)} \\
	n=3 ~  & a_3 = - \frac{1}{4 ~ 3} a_2 = - \frac{1}{4 ~ 3 ~ 3 ~ 2 ~ 2 ~ 1} = \frac{(-1)^3}{(4 ~ 3 ~ 2 ~ 1) ~ (3 ~ 2 ~ 1)}   \\
	\vdots \\
	& a_n =  - \frac{1}{(n+1)n} a_{n-1} = \cdots = \frac{ (-1)^n } { ((n+1) ~ n \cdots 1) ~ (n ~ (n -1) \cdots 1)} = \frac{ (-1)^n } {(n+1)! n!} \\
\ea

Therefore one of the independent solution of the ODE is
\[
	y_1(z) = z \sum_{n=0}^\infty \frac{ (-1)^n } {(n+1)! n!} z^n
\]

\ee

\section*{Question 3}

\section*{Question 4}

\section*{Question 5}

\section*{Question 6}

\end{document}
