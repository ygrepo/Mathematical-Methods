\documentclass[12pt,twoside]{article}
\usepackage[dvipsnames]{xcolor}
\usepackage{tikz,graphicx,amsmath,amsfonts,amscd,amssymb,mathrsfs, bm,cite,epsfig,epsf,url}
\usepackage[hang,flushmargin]{footmisc}
\usepackage[colorlinks=true,urlcolor=blue,citecolor=blue]{hyperref}
\usepackage{amsthm,multirow,wasysym,appendix}
\usepackage{array,subcaption} 
% \usepackage[small,bf]{caption}
\usepackage{bbm}
\usepackage{pgfplots}
\usetikzlibrary{spy}
\usepgfplotslibrary{external}
\usepgfplotslibrary{fillbetween}
\usetikzlibrary{arrows,automata}
\usepackage{thmtools}
\usepackage{blkarray} 
\usepackage{textcomp}
\usepackage[left=0.8in,right=1.0in,top=1.0in,bottom=1.0in]{geometry}


\usepackage{times}
\usepackage{amsfonts}
\usepackage{amsmath}
\usepackage{latexsym}
\usepackage{color}
\usepackage{graphics}
\usepackage{enumerate}
\usepackage{amstext}
\usepackage{blkarray}
\usepackage{url}
\usepackage{epsfig}
\usepackage{bm}
\usepackage{hyperref}
\hypersetup{
    colorlinks=true,
    linkcolor=blue,
    filecolor=magenta,      
    urlcolor=blue,
}
\usepackage{textcomp}
\usepackage[left=0.8in,right=1.0in,top=1.0in,bottom=1.0in]{geometry}
\usepackage{mathtools}
%\usepackage{minted}



\input{macros}

\begin{document}

\noindent Professor Rio\\
EN.585.615.81.SP21 Mathematical Methods\\
Final Exam\\
Johns Hopkins University\\
Student: Yves Greatti\\\

\section*{Question 1}
\be
\item [a.]
$f(x) = x$ is odd on $[-\pi,\pi]$ therefore its Fourier coefficients $a_n$ are $0$ and we need to find its $b_n$ coefficients:
 \ba
 	b_n	&= \frac{2}{2\pi} \int_{-\pi}^{\pi} f(x) \sin(\frac{2 \pi n x}{2 \pi}) dx \\
		&= \frac{4}{2\pi} \int_{0}^{\pi} x \sin(\frac{2 \pi n x}{2 \pi}) dx \\
		&= \frac{2}{\pi} \int_{0}^{\pi} x \sin(n x) dx \\
 \ea
 Using integration by parts:
 \ba
 	 \int_{0}^{\pi} x \sin(n x) dx &= [x (-\frac{\cos(nx)}{n}) ]_0^\pi + \int_{0}^{\pi} 1 \cdot \frac{\cos(nx)}{n} dx \\
	 					&=( -\frac{\pi}{n}) \cos(n \pi) + \frac{1}{n} [\sin(nx)]_0^\pi \\
						&= \frac{(-1)^{n+1} \pi}{n} \\
 \ea
 Thus $b_n = \frac{2}{\pi}   \frac{(-1)^{n+1} \pi}{n} = \frac{(-1)^{n+1} 2}{n}$ and the Fourier series of $x$, on $[-\pi,\pi]$, is:
 \[
 	x = \sum_{n=1}^\infty b_n \sin(nx) = 2  \sum_{n=1}^\infty \frac{(-1)^{n+1}  \sin(nx)}{n}
 \]
 
\item [b.]
If we integrate terms by terms the previous expression, the Fourier series of $x$ over  $[-\pi,\pi]$, we have:
\ba
	\frac{x^2}{2}	&= 2 \sum_{n=1}^\infty \frac{(-1)^{n+1}}{n} (-\frac{\cos(nx)}{n}) + c ~ ~ c\text{constant of integration} \\
			x^2	&= 4 \sum_{n=1}^\infty \frac{(-1)^n}{n^2} \cos(nx) + c \text{ with } 2 c \rightarrow c \\
				&= c + 4 \sum_{n=1}^\infty \frac{(-1)^n}{n^2} \cos(nx)
\ea

\item [c.]
$f(x)=x^2$ is an even function, by Fourier Series for even function over symmetric range, we have:
\[
	x^2 = \frac{a_0}{2} + \sum_{n=1}^\infty a_n \cos{(\frac{2 \pi n x}{2 \pi})} = \frac{a_0}{2}  + \sum_{n=1}^\infty a_n \cos(nx) ~ (1)
\]
where
\ba
	a_0	&= \frac{4}{2\pi} \int_0^\pi x^2 dx \\
		&=  \frac{2}{\pi} [\frac{x^3}{3}]_0^\pi \\
		&= \frac{2}{3} \pi^2 \\
		\\
	a_n 					&= \frac{4}{2\pi} \int_0^\pi x^2 \cos{(\frac{2 \pi n x}{2 \pi})} dx = \frac{2}{\pi} \int_0^\pi x^2 \cos(nx) dx \\
	 \int_0^\pi x^2 \cos(nx) dx 	&= [x^2 ~ \frac{\sin(nx)}{n}]_0^\pi -\frac{2}{n} \int_0^\pi x \sin(nx) dx \\
	 					&= 0 -\frac{2}{n}  \frac{(-1)^{n+1} \pi}{n} \\
	a_n					&=  \frac{2}{\pi} \frac{(-1)^n 2 ~ \pi}{n^2} \\
						&= (-1)^n \frac{4}{n^2} \\
\ea
Substituting for $a_n$ in (1):
\[
	x^2 = \frac{\pi^2}{3} + 4 \sum_{n=1}^\infty \frac{(-1)^n}{n^2} \cos(nx)
\]

\item [d.]
Fourier series of $x^2$ using integration terms by terms or calculating directly match, as required, by taking $c=\frac{\pi^2}{3}$ since $x$ is a piecewise smooth function on the specified range.

\ee

\section*{Question 2}
Consider the differential equation:
\[
	z \frac{d^2y}{dy^2} + y = 0
\]

\be
\item [a.]
We put the equation in standard form:
\[
	\frac{d^2y}{dy^2} + \frac{1}{z} y = 0
\]

$z~p(z)=0$ and $z^2 q(z) = z$ therefore $0$ is a regular singular point.

\item [b.]
Take $y= z^\sigma \sum_{n=0}^\infty a_n z^n$ and the usual derivatives in the D.E. gives by substitution
\ba
	z \sum_{n=0}^\infty  (n+\sigma) (n+\sigma - 1) a_n z^{n+\sigma-2} +  \sum_{n=0}^\infty a_n z^{n+\sigma}	&= 0 \\
	\sum_{n=0}^\infty  (n+\sigma) (n+\sigma - 1) a_n z^{n+\sigma-1} +  \sum_{n=0}^\infty a_n z^{n+\sigma}	&= 0 ~ (1) \\
\ea
Take the term with the lowest power of $z$, which is the first sum with $n=0$, then since each power of $z$ term must be equal to $0$, we have
\[
	\sigma (\sigma-1) a_0 z^{\sigma-1} = 0
\]
Since $a_0 \neq 0$ and $ z^{\sigma-1} \neq 0$, therefore $\sigma =0,1$. 

\item [c.]
We go back to equation $(1)$ and take $\sigma=1$ yields
\[
	\sum_{n=0}^\infty  n (n+1)  a_n z^n +  \sum_{n=0}^\infty a_n z^{n+1} = 0 
\]
Then reindex the second sum to get same power of $z$ in both sums:
\[
	\sum_{n=0}^\infty  n (n+1)  a_n z^n +  \sum_{n=1}^\infty a_{n-1} z^n = 0 
\]
Note, in first term $n = 0$ does not contribute so we can start index at $n = 1$ in the first sum, and combine both sums

\[
	\sum_{n=1}^\infty  [n (n+1)  a_n +  a_{n-1}] z^n = 0
\]
Since every power of $z$ term must be $0$ and $z^n \neq 0$, gives:
\[
	a_n = - \frac{1}{(n+1)n} a_{n-1}
\]
Taking $a_0=1$, now 
\ba
	n =1 ~ & a_1 = - \frac{1}{2 ~ 1} a_0 = - \frac{1}{2 ~ 1} = \frac{(-1)^1}{2 ~ 1} \\
	n =2 ~ & a_2 = - \frac{1}{3 ~ 2} a_1 = \frac{1}{3 ~ 2 ~ 2 ~ 1} = \frac{(-1)^2}{(3 ~ 2 ~ 1) ~ (2 ~ 1)} \\
	n=3 ~  & a_3 = - \frac{1}{4 ~ 3} a_2 = - \frac{1}{4 ~ 3 ~ 3 ~ 2 ~ 2 ~ 1} = \frac{(-1)^3}{(4 ~ 3 ~ 2 ~ 1) ~ (3 ~ 2 ~ 1)}   \\
	\vdots \\
	& a_n =  - \frac{1}{(n+1)n} a_{n-1} = \cdots = \frac{ (-1)^n } { ((n+1) ~ n \cdots 1) ~ (n ~ (n -1) \cdots 1)} = \frac{ (-1)^n } {(n+1)! n!} \\
\ea

Therefore one of the independent solution of the ODE is
\[
	y_1(z) = z \sum_{n=0}^\infty \frac{ (-1)^n } {(n+1)! n!} z^n
\]

\ee

\section*{Question 3}

\section*{Question 4}

\section*{Question 5}

\section*{Question 6}

\end{document}
