\documentclass[12pt,twoside]{article}
\usepackage[dvipsnames]{xcolor}
\usepackage{tikz,graphicx,amsmath,amsfonts,amscd,amssymb,mathrsfs, bm,cite,epsfig,epsf,url}
\usepackage[hang,flushmargin]{footmisc}
\usepackage[colorlinks=true,urlcolor=blue,citecolor=blue]{hyperref}
\usepackage{amsthm,multirow,wasysym,appendix}
\usepackage{array,subcaption} 
% \usepackage[small,bf]{caption}
\usepackage{bbm}
\usepackage{pgfplots}
\usetikzlibrary{spy}
\usepgfplotslibrary{external}
\usepgfplotslibrary{fillbetween}
\usetikzlibrary{arrows,automata}
\usepackage{thmtools}
\usepackage{blkarray} 
\usepackage{textcomp}
\usepackage[left=0.8in,right=1.0in,top=1.0in,bottom=1.0in]{geometry}


\usepackage{times}
\usepackage{amsfonts}
\usepackage{amsmath}
\usepackage{latexsym}
\usepackage{color}
\usepackage{graphics}
\usepackage{enumerate}
\usepackage{amstext}
\usepackage{blkarray}
\usepackage{url}
\usepackage{epsfig}
\usepackage{bm}
\usepackage{hyperref}
\hypersetup{
    colorlinks=true,
    linkcolor=blue,
    filecolor=magenta,      
    urlcolor=blue,
}
\usepackage{textcomp}
\usepackage[left=0.8in,right=1.0in,top=1.0in,bottom=1.0in]{geometry}
\usepackage{mathtools}
\usepackage{minted}
\usepackage{gensymb}


\input{macros}

\begin{document}

\noindent Professor Rio\\
EN.585.615.81.SP21 Mathematical Methods\\
Take Home Project 2\\
Johns Hopkins University\\
Student: Yves Greatti\\\

\section*{Question 1}

\be

\item [(a)]
Please see attached separate pdf.

\item [(b)]

$f(t) = C_0 e^{-\frac{t}{\tau}}$ with period $T$, so
\ba
	a_0	&= \frac{2}{T} \int_0^T C_0 e^{-\frac{t}{\tau}} dt \\
		&= \frac{2 C_0}{T} (-\tau) [e^{-\frac{t}{\tau}}]_0^T \\
		&= -2 C_0 \frac{\tau}{T} [ e^{-\frac{T}{\tau}} - 1] \\
		&= 2 C_0 \frac{\tau}{T} (1 - e^{-\frac{T}{\tau}}) \\
\ea
If $\tau \ll T$ then $e^{-\frac{T}{\tau}} \approx 0$ and $a_0 \approx 2 C_0 \frac{\tau}{T}$.
\ba
	a_k	&=  \frac{2}{T} \int_0^T C_0 e^{-\frac{t}{\tau}} \cos{\frac{2 k \pi t}{T}}dt \\
		&=   \frac{2 C_0}{T} \int_0^T e^{-\frac{t}{\tau}} \cos{\frac{2 k \pi t}{T}}dt \\
\ea
Using integration by parts with $u= \cos{\frac{2 k \pi t}{T}}, du = - \frac{2 k \pi}{T} \sin{\frac{2 k \pi t}{T}}$ and $dv =  e^{-\frac{t}{\tau}}, v = (-\tau)  e^{-\frac{t}{\tau}}$:
\[
	\int_0^T e^{-\frac{t}{\tau}} \cos{\frac{2 k \pi t}{T}} dt	=   (-\tau)  [e^{-\frac{t}{\tau}} \cos{\frac{2 k \pi t}{T}}]_0^T - \frac{2 k \pi \tau}{T} \int_0^T e^{-\frac{t}{\tau}} \sin{\frac{2 k \pi t}{T}}  dt \\
\]
Using again integration by parts:
\[
 	\int_0^T e^{-\frac{t}{\tau}} \sin{\frac{2 k \pi t}{T}}  dt =  (-\tau)  [e^{-\frac{t}{\tau}} \sin{\frac{2 k \pi t}{T}}]_0^T + \frac{2 k \pi \tau}{T} \int_0^T e^{-\frac{t}{\tau}} \cos{\frac{2 k \pi t}{T}}  dt 
\]
So
\ba
	(1 + (\frac{2 k \pi \tau}{T}))^2 \int_0^T e^{-\frac{t}{\tau}} \cos{\frac{2 k \pi t}{T}} dt &= (-\tau)  [e^{-\frac{t}{\tau}} \cos{\frac{2 k \pi t}{T}}]_0^T  + \frac{2 k \pi \tau^2}{T}   [e^{-\frac{t}{\tau}} \sin{\frac{2 k \pi t}{T}}]_0^T  \\
																&= (-\tau)  [e^{-\frac{t}{\tau}} \cos{\frac{2 k \pi t}{T}}]_0^T  + 0 \\
																&= \tau (1 - e^{-\frac{T}{\tau}} ) \\
						 \int_0^T e^{-\frac{t}{\tau}} \cos{\frac{2 k \pi t}{T}} dt	&=  \frac{\tau} {1 + (\frac{2 k \pi \tau}{T})^2}  (1 - e^{-\frac{T}{\tau}} ) \\																
\ea
Substituting back into the expression found for $a_k$ yields
\ba
	a_k 	&= 2 C_0 \frac{\tau} {T} \frac{1} {1 + (\frac{2 k \pi \tau}{T})^2} (1 - e^{-\frac{T}{\tau}} ) \\
		&=  2 C_0 \frac{\tau T} {T^2 + (2 k \pi \tau)^2}  (1 - e^{-\frac{T}{\tau}} ) \\
\ea
With the same assumption $\tau \ll T$ then $e^{-\frac{T}{\tau}} \approx 0$ and $a_k \approx 2 C_0 \frac{\tau} {T} \frac{1} {1 + (\frac{2 k \pi \tau}{T})^2}$.
Similarly to compute $b_k$
\ba
	b_k	&=  \frac{2}{T} \int_0^T C_0 e^{-\frac{t}{\tau}} \sin{\frac{2 k \pi t}{T}}dt \\
		&=   \frac{2 C_0}{T} \int_0^T e^{-\frac{t}{\tau}} \sin{\frac{2 k \pi t}{T}}dt \\
		&=   \frac{2 C_0}{T} \frac{2 k \pi \tau}{T} \int_0^T e^{-\frac{t}{\tau}} \cos{\frac{2 k \pi t}{T}}  dt \\
		&=   \frac{2 C_0}{T} \frac{2 k \pi \tau}{T} \frac{\tau} {1 + (\frac{2 k \pi \tau}{T})^2}  (1 - e^{-\frac{T}{\tau}} ) \\
		&=	4 C_0 k \pi \frac{\tau^2} {T^2 + (2 k \pi \tau)^2}  (1 - e^{-\frac{T}{\tau}} ) \\
\ea
Once again, since $e^{-\frac{T}{\tau}} \approx 0$ and $b_k \approx  	4 C_0 (\frac{\tau} {T})^2  \frac{1} {1 + (\frac{2 k \pi \tau}{T})^2}  \pi k$

\item [(c)]

For $k \ge 1$

\ba
	p_k	&= \frac{1}{2} (a_k^2 + b_k^2) \\
		&= \frac{1}{2} \bigg [ 4 C_0^2 (\frac{\tau} {T})^2  \frac{1} {(1 + (\frac{2 k \pi \tau}{T})^2)^2} + 
				16 C_0^2 (\frac{\tau} {T})^4   \frac{1} {(1 + (\frac{2 k \pi \tau}{T})^2)^2}  \pi^2 k^2 \bigg ] \\
		&= \frac{1}{2} 4 C_0^2  (\frac{\tau} {T})^2  \frac{1} {(1 + (\frac{2 k \pi \tau}{T})^2)^2}   \bigg [ 1 + 4  (\frac{\tau} {T})^2  \pi^2 k^2 \bigg ] \\			
		&= 2 C_0^2  (\frac{\tau} {T})^2  \frac{1} {(1 + (\frac{2 k \pi \tau}{T})^2)^2}   \bigg [ 1 + 4  (\frac{\tau} {T})^2  \pi^2 k^2 \bigg ] \\			
\ea

\item [(d)]

\item [(e)]

\item [(f)]
 We have
 \ba
 	a_k \cos(\frac{k 2 \pi t}{T}) + b_k \sin(\frac{k 2 \pi t}{T}) 	&= \cos(\phi_k) \cos(\frac{k 2 \pi t}{T}) + \sin(\phi_k) \sin(\frac{k 2 \pi t}{T}) \\
												&= \cos(\frac{k 2 \pi t}{T} - \phi_k) \\
 \ea
 where
 \ba
 	\tan(\phi_k) = \frac{\sin(\phi_k)} {\cos(\phi_k)} = \frac{b_k}{a_k}	&= 4 C_0 (\frac{\tau} {T})^2  \frac{1} {1 + (\frac{2 k \pi \tau}{T})^2}  \pi k  (2 C_0 \frac{\tau} {T} \frac{1} {1 + (\frac{2 k \pi \tau}{T})^2})^{-1} \\
													&= 2 \frac{\tau}{T} \pi k \\
											\phi_k	&= \arctan(2 \frac{\tau}{T} \pi k) \\
 \ea
 For $\frac{\tau}{T} = .1$, $\phi_1 \approx 32.14^{\circ}$  and $\phi_2 \approx 51.48^{\circ}$
 and for $\frac{\tau}{T} = .01$, $\phi_1 \approx 3.59^{\circ}$  and $\phi_2 \approx 7.16^{\circ}$


\ee

\section*{Question 2}

\be
\item [(a)]
One simple way to describe $P(r)$ is to define it as $P(r) = A r + B$ with the conditions:
\ba
	A \cdot 0 + B &= Q \\
	A \cdot R + B &= 0 \\	
\ea
which gives $A=-\frac{Q}{R}$ and $B=Q$.
So
\ba
	P(r) =
	\begin{cases}
	Q (1 -\frac{r}{R}) ~ \text{ for } 0 \le r \le R\ \\
	0 ~ \text{ for }  r > R \\
	\end{cases} \\	
\ea


\begin{figure}[H]
		\centering
		\includegraphics[width=100pt]{pr}
		\caption{$P(r)$}
		\label{fig1}
\end{figure}
	

\item [(b)]

Since we assume no angular dependence: $\nabla^2 C = \frac {1}{r^2} \frac{d}{d r}(r^2 \frac{d C}{d r})$,  the differential equation is now:
\ba
	\frac{D}{r^2} \frac{d}{d r}(r^2 \frac{d C(r)}{d r}) +P(r)	&= 0 \\
	\frac{d}{d r}(r^2 \frac{d C(r)}{d r})				&= - \frac{r^2}{D} P(r) \\
\ea

\item [(c)]
Inside the cell $P(r) = Q (1 -\frac{r}{R})$, so we have to solve the differential equation
\ba
				\frac{d}{d r}(r^2 \frac{d C(r)}{d r})	&=  - \frac{r^2}{D} Q (1 -\frac{r}{R}) \\
											&=  \frac{Q}{D R } r^2 (r-R) \\
											&= \frac{Q}{D R } r^3 - \frac{Q}{D} r^2 \\
\ea
Integrating once
\ba
	r^2 \frac{d C(r)}{d r}	&=  \frac{Q}{4 D R } r^4 - \frac{Q}{3 D} r^3 + A  \\
	 \frac{d C(r)}{d r}	&=  \frac{Q}{4 D R } r^2 - \frac{Q}{3 D} r + \frac{A}{r^2} \\
\ea
Integrating again
\[
	 C_i(r)	=  \frac{Q}{12 D R } r^3 - \frac{Q}{6 D} r^2 - \frac{A}{r} + B ~ ~ A,B \text{:constants}, C_i \text{:inside cell concentration}
\]
Outside the cell $P(r)=0$ and the we want to solve the differential equation
\[
				\frac{d}{d r}(r^2 \frac{d C(r)}{d r}) =  0
\]
Which by integration gives
\ba
	r^2 \frac{d C(r)}{d r} 		&= C_1 \\
	\frac{d C(r)}{d r}			&= \frac{C_1}{r^2} \\
	C_o(r)				&= -\frac{C_1}{r} + C_2 ~ ~ C_1,C_2 \text{:constants}, C_o \text{:outside cell concentration} \\
\ea

\item [(d)]

Applying the boundary conditions
\be
	\item [(i)]
	\[
		\lim_{r \rightarrow 0} C_i(r) = \lim_{r \rightarrow 0} \frac{Q}{12 D R } r^3 - \frac{Q}{6 D} r^2 - \frac{A}{r} + B
	\]
	since $\lim_{r \rightarrow 0} C_i(r) = \frac{1}{r} = \infty$ therefore to have finite concentration $C_i(r)$ at $r=0$ we need $A=0$
	
	\item [(ii)]
	\[
		\lim_{r \rightarrow \infty} C_o(r) = \lim_{r \rightarrow \infty} \bigg ( -\frac{C_1}{r} + C_2 \bigg ) = C_2
	\]
	The concentration goes to zero at infinity implies $C_2=0$
	
	\item [(iii)]
	We have now for $C_i(r)$ and $C_o(r)$:
	\ba
		C_i(r)	&=  \frac{Q}{12 D R } r^3 - \frac{Q}{6 D} r^2 + B \\
		C_o(r)	&= -\frac{C_1}{r}
	\ea
	$C_i(R) = C_o(R)$ and $\frac{d C_i(r)}{dr} = \frac{d C_o(r)}{dr} |_{r=R}$ yields
	\ba
		 \frac{Q}{12 D R } R^3 - \frac{Q}{6 D} R^2 + B 	&= -\frac{C_1}{R}\\
		 \frac{Q}{4 D} R - \frac{Q}{3 D} R 		&= \frac{C_1}{R^2}\\
	\ea
	Rearranging
	\ba
		 -\frac{Q}{12 D} R^2 + B 	&= -\frac{C_1}{R}\\
		 - \frac{Q}{12 D} R 		&= \frac{C_1}{R^2}\\
	\ea
	which gives
	\ba
		B 	&= \frac{Q}{6 D} R^2 \\
		C_1  &= -\frac{Q}{12 D} R^3 \\
	\ea	
	
substituting back
\ba
	C_i(r)	&=  \frac{Q}{12 D R } r^3 - \frac{Q}{6 D} r^2 +  \frac{Q}{6 D} R^2 \\
	C_o(r)	&= \frac{Q}{12 D} R^3 \frac{1}{r} \\
\ea
	
\ee

\item [(e)]
The concentration maximum happens within the cell since $P(r)$ has maximum value $Q$ at $r=0$ and then it is zero for $r>R$.
We are looking for the value of $r$ for which $\frac{d C_i(r)}{d r}=0$:
\[
	\frac{d C_i(r)}{d r} = \frac{Q}{4 D R} r^2 - \frac{Q}{3 D} r = \frac{Q}{D} r (\frac{r}{4 R} - \frac{1}{3} )
\]
Discarding the solution $r=0$ we are left that  concentration maximum is for $r=\frac{4}{3} R$ and it is
\ba
	C_M 	&= \frac{Q}{12 D R }  (\frac{4}{3})^3 R^3 - \frac{Q}{6 D}  (\frac{4}{3})^2 R^2  +  \frac{Q}{6 D} R^2 \\
			&= \frac{Q}{6D} R^2 \bigg [ \frac{4^3}{2 ~ 3^3} - \frac{4^2}{3^2} + 1 \bigg ] \\
			&= \frac{11}{162} \frac{Q}{D} R^2 \\
\ea
Inside the cell
\ba
	C_i(r)			&=  \frac{Q}{6 D} (\frac{1}{2} \frac{r^3}{R} - r^2 + R^2) \\
	\frac{C_i(r)}{C_M}	&= \frac{Q}{6 D}   \frac{162}{11}  \frac{D}{Q} R^{-2} (\frac{1}{2} \frac{r^3}{R} - r^2 + R^2) \\
					&= \frac{162}{6 ~ ~ 11} \bigg [ \frac{1}{2}  (\frac{r}{R})^3 -  (\frac{r}{R})^2 + 1 \bigg ] \\
\ea
	
\ee

\end{document}
