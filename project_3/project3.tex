\documentclass[12pt,twoside]{article}
\usepackage[dvipsnames]{xcolor}
\usepackage{tikz,graphicx,amsmath,amsfonts,amscd,amssymb,mathrsfs, bm,cite,epsfig,epsf,url}
\usepackage[hang,flushmargin]{footmisc}
\usepackage[colorlinks=true,urlcolor=blue,citecolor=blue]{hyperref}
\usepackage{amsthm,multirow,wasysym,appendix}
\usepackage{array,subcaption} 
% \usepackage[small,bf]{caption}
\usepackage{bbm}
\usepackage{pgfplots}
\usetikzlibrary{spy}
\usepgfplotslibrary{external}
\usepgfplotslibrary{fillbetween}
\usetikzlibrary{arrows,automata}
\usepackage{thmtools}
\usepackage{blkarray} 
\usepackage{textcomp}
\usepackage{float}
%\usepackage[left=0.8in,right=1.0in,top=1.0in,bottom=1.0in]{geometry}


\usepackage{times}
\usepackage{amsfonts}
\usepackage{amsmath}
\usepackage{latexsym}
\usepackage{color}
\usepackage{graphics}
\usepackage{enumerate}
\usepackage{amstext}
\usepackage{blkarray}
\usepackage{url}
\usepackage{epsfig}
\usepackage{bm}
\usepackage{hyperref}
\hypersetup{
    colorlinks=true,
    linkcolor=blue,
    filecolor=magenta,      
    urlcolor=blue,
}
\usepackage{textcomp}
%\usepackage[left=0.8in,right=1.0in,top=1.0in,bottom=1.0in]{geometry}
\usepackage{mathtools}
%\usepackage{minted}
\usepackage{gensymb}

%% Probability operators and functions
%
% \def \P{\mathrm{P}}
\def \P{\mathrm{P}}
\def \E{\mathrm{E}}
\def \Var{\mathrm{Var}}
\let\var\Var
\def \Cov {\mathrm{Cov}} \let\cov\Cov
\def \MSE {\mathrm{MSE}} \let\mse\MSE
\def \sgn {\mathrm{sgn}}
\def \R {\mathbb{R}}
\def \C {\mathbb{C}}
\def \N {\mathbb{N}}
\def \Z {\mathbb{Z}}
\def \cV {\mathcal{V}}
\def \cS {\mathcal{S}}

\newcommand{\RR}{\ensuremath{\mathbb{R}}}

\DeclareMathOperator*{\argmin}{arg\,min}
\DeclareMathOperator*{\argmax}{arg\,max}
\newcommand{\red}[1]{\textcolor{red}{#1}}
\newcommand{\blue}[1]{\textcolor{blue}{#1}}
\newcommand{\green}[1]{\textcolor{ForestGreen}{ #1}}
\newcommand{\fuchsia}[1]{\textcolor{RoyalPurple}{ #1}}



%
%% Probability distributions
%
%\def \Bern    {\mathrm{Bern}}
%\def \Binom   {\mathrm{Binom}}
%\def \Exp     {\mathrm{Exp}}
%\def \Geom    {\mathrm{Geom}}
% \def \Norm    {\mathcal{N}}
%\def \Poisson {\mathrm{Poisson}}
%\def \Unif    {\mathrm {U}}
%
\DeclareMathOperator{\Norm}{\mathcal{N}}

\newcommand{\bdb}[1]{\textcolor{red}{#1}}

\newcommand{\ml}[1]{\mathcal{ #1 } }
\newcommand{\wh}[1]{\widehat{ #1 } }
\newcommand{\wt}[1]{\widetilde{ #1 } }
\newcommand{\conj}[1]{\overline{ #1 } }
\newcommand{\rnd}[1]{\tilde{ #1 } }
\newcommand{\rv}[1]{ \rnd{ #1}  }
\newcommand{\rx}{\rnd{ x}  }
\newcommand{\ry}{\rnd{ y}  }
\newcommand{\rz}{\rnd{ z}  }
\newcommand{\ra}{\rnd{ a}  }
\newcommand{\rb}{\rnd{ b}  }
\newcommand{\rpc}{\widetilde{ pc}  }
\newcommand{\rndvec}[1]{\vec{\rnd{#1}}}

\def \cnd {\, | \,}
\def \Id { I }
\def \J {\mathbf{1}\mathbf{1}^T}

\newcommand{\op}[1]{\operatorname{#1}}
\newcommand{\setdef}[2]{ := \keys{ #1 \; | \; #2 } }
%\newcommand{\set}[2]{ \keys{ #1 \; | \; #2 } }
\newcommand{\sign}[1]{\op{sign}\left( #1 \right) }
\newcommand{\trace}[1]{\op{tr}\left( #1 \right) }
\newcommand{\tr}[1]{\op{tr}\left( #1 \right) }
\newcommand{\inv}[1]{\left( #1 \right)^{-1} }
%\newcommand{\abs}[1]{\left| #1 \right|}
\newcommand{\sabs}[1]{| #1 |}
\newcommand{\keys}[1]{\left\{ #1 \right\}}
\newcommand{\sqbr}[1]{\left[ #1 \right]}
\newcommand{\sbrac}[1]{ ( #1 ) }
\newcommand{\brac}[1]{\left( #1 \right) }
\newcommand{\bbrac}[1]{\big( #1 \big) }
\newcommand{\Bbrac}[1]{\Big( #1 \Big)}
\newcommand{\BBbrac}[1]{\BIG( #1 \Big)}
\newcommand{\MAT}[1]{\begin{bmatrix} #1 \end{bmatrix}}
\newcommand{\sMAT}[1]{\left(\begin{smallmatrix} #1 \end{smallmatrix}\right)}
\newcommand{\sMATn}[1]{\begin{smallmatrix} #1 \end{smallmatrix}}
\newcommand{\PROD}[2]{\left \langle #1, #2\right \rangle}
\newcommand{\PRODs}[2]{\langle #1, #2 \rangle}
\newcommand{\der}[2]{\frac{\text{d}#2}{\text{d}#1}}
\newcommand{\pder}[2]{\frac{\partial#2}{\partial#1}}
\newcommand{\derTwo}[2]{\frac{\text{d}^2#2}{\text{d}#1^2}}
\newcommand{\ceil}[1]{\lceil #1 \rceil}
\newcommand{\Imag}[1]{\op{Im}\brac{ #1 }}
\newcommand{\Real}[1]{\op{Re}\brac{ #1 }}
%\newcommand{\norm}[1]{\left|\left| #1 \right|\right| }
\newcommand{\norms}[1]{ \| #1 \|  }
\newcommand{\normProd}[1]{\left|\left| #1 \right|\right| _{\PROD{\cdot}{\cdot}} }
\newcommand{\normTwo}[1]{\left|\left| #1 \right|\right| _{2} }
\newcommand{\normTwos}[1]{ \| #1  \| _{2} }
\newcommand{\normZero}[1]{\left|\left| #1 \right|\right| _{0} }
\newcommand{\normTV}[1]{\left|\left| #1 \right|\right|  _{ \op{TV}  } }% _{\op{c} \ell_1} }
\newcommand{\normOne}[1]{\left|\left| #1 \right|\right| _{1} }
\newcommand{\normOnes}[1]{\| #1 \| _{1} }
\newcommand{\normOneTwo}[1]{\left|\left| #1 \right|\right| _{1,2} }
\newcommand{\normF}[1]{\left|\left| #1 \right|\right| _{\op{F}} }
\newcommand{\normLTwo}[1]{\left|\left| #1 \right|\right| _{\ml{L}_2} }
\newcommand{\normNuc}[1]{\left|\left| #1 \right|\right| _{\ast} }
\newcommand{\normOp}[1]{\left|\left| #1 \right|\right|  }
\newcommand{\normInf}[1]{\left|\left| #1 \right|\right| _{\infty}  }
\newcommand{\proj}[1]{\mathcal{P}_{#1} \, }
\newcommand{\diff}[1]{ \, \text{d}#1 }
\newcommand{\vc}[1]{\boldsymbol{\vec{#1}}}
\newcommand{\rc}[1]{\boldsymbol{#1}}
\newcommand{\vx}{\vec{x}}
\newcommand{\vy}{\vec{y}}
\newcommand{\vz}{\vec{z}}
\newcommand{\vu}{\vec{u}}
\newcommand{\vv}{\vec{v}}
\newcommand{\vb}{\vec{\beta}}
\newcommand{\va}{\vec{\alpha}}
\newcommand{\vaa}{\vec{a}}
\newcommand{\vbb}{\vec{b}}
\newcommand{\vg}{\vec{g}}
\newcommand{\vw}{\vec{w}}
\newcommand{\vh}{\vec{h}}
\newcommand{\vbeta}{\vec{\beta}}
\newcommand{\valpha}{\vec{\alpha}}
\newcommand{\vgamma}{\vec{\gamma}}
\newcommand{\veta}{\vec{\eta}}
\newcommand{\vnu}{\vec{\nu}}
\newcommand{\rw}{\rnd{w}}
\newcommand{\rvnu}{\vc{\nu}}
\newcommand{\rvv}{\rndvec{v}}
\newcommand{\rvw}{\rndvec{w}}
\newcommand{\rvx}{\rndvec{x}}
\newcommand{\rvy}{\rndvec{y}}
\newcommand{\rvz}{\rndvec{z}}
\newcommand{\rvX}{\rndvec{X}}


\newtheorem{theorem}{Theorem}[section]
% \declaretheorem[style=plain,qed=$\square$]{theorem}
\newtheorem{corollary}[theorem]{Corollary}
\newtheorem{definition}[theorem]{Definition}
\newtheorem{lemma}[theorem]{Lemma}
\newtheorem{remark}[theorem]{Remark}
\newtheorem{algorithm}[theorem]{Algorithm}

% \theoremstyle{definition}
%\newtheorem{example}[proof]{Example}
\declaretheorem[style=definition,qed=$\triangle$,sibling=definition]{example}
\declaretheorem[style=definition,qed=$\bigcirc$,sibling=definition]{application}

%
%% Typographic tweaks and miscellaneous
%\newcommand{\sfrac}[2]{\mbox{\small$\displaystyle\frac{#1}{#2}$}}
%\newcommand{\suchthat}{\kern0.1em{:}\kern0.3em}
%\newcommand{\qqquad}{\kern3em}
%\newcommand{\cond}{\,|\,}
%\def\Matlab{\textsc{Matlab}}
%\newcommand{\displayskip}[1]{\abovedisplayskip #1\belowdisplayskip #1}
%\newcommand{\term}[1]{\emph{#1}}
%\renewcommand{\implies}{\;\Rightarrow\;}

% My macros

\def\Kset{\mathbb{K}}
\def\Nset{\mathbb{N}}
\def\Qset{\mathbb{Q}}
\def\Rset{\mathbb{R}}
\def\Sset{\mathbb{S}}
\def\Zset{\mathbb{Z}}
\def\squareforqed{\hbox{\rlap{$\sqcap$}$\sqcup$}}
\def\qed{\ifmmode\squareforqed\else{\unskip\nobreak\hfil
\penalty50\hskip1em\null\nobreak\hfil\squareforqed
\parfillskip=0pt\finalhyphendemerits=0\endgraf}\fi}

%\DeclareMathOperator*{\E}{\rm E}
%\DeclareMathOperator*{\argmax}{\rm argmax}
%\DeclareMathOperator*{\argmin}{\rm argmin}
%\DeclareMathOperator{\sgn}{sign}
\DeclareMathOperator{\supp}{supp}
\DeclareMathOperator{\last}{last}
%\DeclareMathOperator{\sign}{\sgn}
\DeclareMathOperator{\diag}{diag}
\providecommand{\abs}[1]{\lvert#1\rvert}
\providecommand{\norm}[1]{\lVert#1\rVert}
\def\vcdim{\textnormal{VCdim}}
\DeclareMathOperator*{\B}{\textbf{B}}

%\DeclarePairedDelimiter\ceil{\lceil}{\rceil}
%\DeclarePairedDelimiter\floor{\lfloor}{\rfloor}

\newcommand{\cX}{{\mathcal X}}
\newcommand{\cY}{{\mathcal Y}}
\newcommand{\cA}{{\mathcal A}}
\newcommand{\ignore}[1]{}
\newcommand{\ba}{\[\begin{aligned}}
\newcommand{\ea}{\end{aligned}\]}
\newcommand{\bi}{\begin{itemize}}
\newcommand{\ei}{\end{itemize}}
\newcommand{\be}{\begin{enumerate}}
\newcommand{\ee}{\end{enumerate}}
\newcommand{\bd}{\begin{description}}
\newcommand{\ed}{\end{description}}
\newcommand{\h}{\widehat}
\newcommand{\e}{\epsilon}
\newcommand{\mat}[1]{{\mathbf #1}}
%\newcommand{\R}{\mat{R}}
\newcommand{\0}{\mat{0}}
\newcommand{\M}{\mat{M}}

\newcommand{\D}{\mat{D}}
\renewcommand{\r}{\mat{r}}
\newcommand{\x}{\mat{x}}
\renewcommand{\u}{\mat{u}}
\renewcommand{\v}{\mat{v}}
\newcommand{\w}{\mat{w}}
\renewcommand{\H}{\text{0}}
\newcommand{\T}{\text{1}}
%\newcommand{\set}[1]{\{#1\}}
\newcommand{\xxi}{{\boldsymbol \xi}}
\newcommand{\ssigma}{{\boldsymbol \sigma}}
\newcommand{\Alpha}{{\boldsymbol \alpha}}
\newcommand{\tts}{\tt \small}
\newcommand{\hint}{\emph{hint}}
\newcommand{\matr}[1]{\bm{#1}}     % ISO complying version
\newcommand{\vect}[1]{\bm{#1}} % vectors

%\newcommand{\Var}{\mathrm{Var}}
%\newcommand{\Cov}{\mathrm{Cov}}

% New commands
\newcommand{\SP}{\mathbf{S}_{+}^n}
\newcommand{\Proj}{\mathcal{P}_{\mathcal{S}}}
%\DeclarePairedDelimiterX{\inp}[2]{\langle}{\rangle}{#1, #2}




\begin{document}

\noindent Professor Rio\\
EN.585.615.81.SP21 Mathematical Methods\\
Take Home Project 3\\
Johns Hopkins University\\
Student: Yves Greatti\\\

\section*{Question 1}
The Wormersley equation for blood flow is:
\[
	\rho \frac{\partial w}{\partial t} = \frac{\mu}{r} \frac{\partial}{\partial r}( r \frac{\partial w}{\partial r}) +\frac{\partial P}{\partial z} 
\]
Using $\frac{\partial P}{\partial z}  = A e^{int}$ and taking $w(r, t) = u(r) e^{int}$ yields: $\frac{\partial w}{\partial t} = (in) u e^{int}$,  
$\frac{\partial w}{\partial r} = u'(r) e^{int}$, and $\frac{\partial^2 w}{\partial r^2} = u''(r) e^{int}$, 
$\frac{\partial}{\partial r}( r \frac{\partial w}{\partial r}) = u'(r) e^{int} + r u''(r) e^{int}$.
Therefore the Wormersley equation becomes:
\ba
	\frac{\mu}{r} \bigg [ u'(r) e^{int} + r u''(r) e^{int} \bigg ]	+  A e^{int}		&=	\rho (i~n) u(r) e^{int} \\
	\mu \frac{d^2 u(r)}{d r^2} + \frac{\mu}{r} \frac{d u(r)}{d r} + A 			&= (i~n )~ \rho ~ u(r) \text{ by dividing through } e^{int} \\
	\frac{d^2 u(r)}{d r^2} + \frac{1}{r} \frac{d u(r)}{d r} - \frac{i~n~\rho}{\mu} u	&= - \frac{A}{\mu} \text{ by dividing through } \mu \text{ and rearranging} \\
\ea
Finally using $\nu = \frac{\mu}{\rho}$ we have:
\[
	\frac{d^2 u(r)}{d r^2} + \frac{1}{r} \frac{d u(r)}{d r} - \frac{i~n}{\nu} u = - \frac{A}{\mu}
\]
By simple inspection, one particular solution is a constant w.r.t. $r$, such as $u_p = C$, substituting it into the differential equation gives:
\[
	- \frac{i~n~\rho}{\mu} u_p = - \frac{A}{\mu}
\]
thus $u_p = \frac{A}{in\rho}$.
The homogeneous equation is:
\[
	\frac{d^2 u(r)}{d r^2} + \frac{1}{r} \frac{d u(r)}{d r} + \frac{i^3~n}{\nu} u = 0
\]
Take $\lambda^2 = \frac{i^3~n}{\nu}$, we now have:

\ba
	\frac{d^2 u(r)}{d r^2} + \frac{1}{r} \frac{d u(r)}{d r} + \lambda^2 u	&= 0 \\
	r^2 \frac{d^2 u(r)}{d r^2} + r \frac{d u(r)}{d r} + (\lambda r)^2 u	&= 0 ~ ~ (1)\\
\ea
Take $x=\lambda r$, then:
\ba
	\frac{d u(x)}{dr} &= \frac{d u(\lambda r)}{dr}=\lambda   \frac{d u(x)}{dx} \\
	\frac{d^2 u(x)}{dr^2} &= \lambda^2   \frac{d^2 u(x)}{dx^2} \\
\ea
Substitute back into (1), we have
\ba
	\lambda^2 r^2 \frac{d^2 u(x)}{dx^2} + \lambda r \frac{d u(x)}{d x} + (\lambda r)^2 u(x)	&= 0 \\
	x^2 \frac{d^2 u(x)}{dx^2} + x  \frac{d u(x)}{d x} +  x^2 u						&= 0 \\
\ea
The last equation is a Bessel's equation of order $0$, therefore
the solution, $u_h$, of the homogeneous equation is a solution of a Bessel's equation of order $0$:
\[
	u_h(r) = C_1 J_0(\lambda r) + C_2 Y_0(\lambda r)
\]
And
\[
	u(r) = u_h(r) + u_p(r) = C_1 J_0(\lambda r) + C_2 Y_0(\lambda r)  +\frac{A}{in\rho}
\]
Now we apply the boundary conditions to our solution.
\[
	u'(r) = C_1 J'_0(\lambda r) + C_2 Y'_0(\lambda r)
\]
We have
\ba
	J_0(x)		&= \sum_{n=0}^\infty	\frac{(-1)^n x^{2n}} {2^{2n} n! \Gamma(1+n)} \\
				&= 1 - \frac{x^2}{2^2} + \frac{x^4}{2^2 4^2} - \cdots \\
	J'_0(x)		&= -2 \frac{x}{2^2} + 4 \frac{x^3}{2^2 4^2} - \cdots \\
	J'_0(0)		&= 0 \\
	\lim_{r \rightarrow 0} u'(r) &= \lim_{r \rightarrow 0} C_1 J'_0(\lambda r) + C_2 Y'_0(\lambda r) \\
					&=  0 + \lim_{r \rightarrow 0} C_2 Y'_0(\lambda r) \\
\ea
Looking at the plot of $Y_0(x)$, we see that in order to have $\frac{\partial w} {\partial r} |_{r=0} = 0$ or $\frac{\partial u} {\partial r} |_{r=0} = 0$
, the term in $Y_0$ must be discarded and we need $C_2=0$.
Thus
\[
	u(r) =   C_1 J_0(\lambda r) +\frac{A}{i~n~\rho}
\]
Using the second boundary condition $w(R) = u(R) = 0$ we have $C_1 J_0(\lambda R)+\frac{A}{i~n~\rho} = 0$ or $C_1 = -\frac{A}{i ~n~\rho J_0(\lambda R)}$ 
Putting everything back
\ba
	u(r)	&= \frac{A}{\rho~i~n} \bigg [ 1 - \frac{J_0(\lambda r)} {J_0(\lambda R)} \bigg ] \\
		&= \frac{A}{\rho~i~n} \bigg [ 1 - \frac{J_0(r \sqrt{\frac{\lambda}{\nu}} i^{\frac{3}{2}})} {J_0(R \sqrt{\frac{\lambda}{\nu}} i^{\frac{3}{2}}) } \bigg ] \\
\ea
Take $\alpha = R \sqrt{\frac{\lambda}{\nu}}$ and $y=\frac{r}{R}$ then
\ba
	J_0(r \sqrt{\frac{\lambda}{\nu}} i^{\frac{3}{2}}) &= J_0(\frac{r}{R} R \sqrt{\frac{\lambda}{\nu}} i^{\frac{3}{2}}) = J_0(\alpha y  i^{\frac{3}{2}}) \\
	J_0(R \sqrt{\frac{\lambda}{\nu}} i^{\frac{3}{2}}) &=  J_0(\alpha i^{\frac{3}{2}}) \\
\ea
Lastly
\[
	w(y,t) = u(r) e^{int} = \frac{A}{\rho~i~n} \bigg [ 1 - \frac{ J_0(\alpha y  i^{\frac{3}{2}}) } { J_0(\alpha i^{\frac{3}{2}}) } \bigg ] e^{int}
\]

\section*{Question 2}

From 
\[
	Q = 2 \pi \int_0^R w(r,t) r dr
\]
Make the change of variable $y = \frac{r}{R}, dy = \frac{dr}{R}$ and we have
\[
	Q = 2 \pi  \int_0^1 w(y,t) R^2 y ~ dy = 2 \pi R^2 \int_0^1 w ~ y ~ dy
\]
Plugging the expression of $w$ found in the previous question
\ba
	Q 	&= 2 \pi R^2 \frac{A}{\rho~i~n}  \int_0^1   \bigg [ 1 - \frac{ J_0(\alpha y  i^{\frac{3}{2}}) } { J_0(\alpha i^{\frac{3}{2}}) } \bigg ] e^{int} ~ y ~ dy \\
		&=  \frac{2 \pi R^2  A}{\rho~i~n} e^{int}  \bigg [ \int_0^1 y~dy - \frac{1}{  J_0(\alpha i^{\frac{3}{2}})  }   \int_0^1 y J_0(\alpha y  i^{\frac{3}{2}}) ~ dy \bigg ] \\
\ea
$\int_0^1 y~dy = [ \frac{y^2}{2} ]_0^1 = \frac{1}{2}$ and we make the change of variable $s=\alpha  i^{\frac{3}{2}} y, ds = \alpha  i^{\frac{3}{2}} dy$ so
\ba
	 \int_0^1 y J_0(\alpha y  i^{\frac{3}{2}}) ~ dy	&= \int_0^{\alpha  i^{\frac{3}{2}} } \frac{s}{\alpha  i^{\frac{3}{2}}} J_0(s) ~ \frac{1}{\alpha  i^{\frac{3}{2}}} ~ ds \\
	 									&= \frac{1}{\alpha^2 i^3}  \int_0^{\alpha  i^{\frac{3}{2}} } s J_0(s) ~ ds \\
	 									&= \frac{\alpha  i^{\frac{3}{2}}}{\alpha^2 i^3} J_1(\alpha  i^{\frac{3}{2}})\\
\ea
Therefore
\ba
	Q 	&= \frac{2 \pi R^2  A}{\rho~i~n} e^{int} \bigg [  \frac{1}{2} -  \frac{\alpha  i^{\frac{3}{2}}}{\alpha^2 i^3} \frac { J_1(\alpha  i^{\frac{3}{2}}) } {  J_0(\alpha i^{\frac{3}{2}})  }  \bigg ] \\
		&=  \frac{\pi R^2 }{\rho} \frac{A}{i~n}  \bigg [ 1 - \frac{2 \alpha  i^{\frac{3}{2}} } { i^3 \alpha^2}  \frac { J_1(\alpha  i^{\frac{3}{2}}) } {  J_0(\alpha i^{\frac{3}{2}})  }  \bigg ] e^{int}   \\
\ea

\section*{Question 3}
Start with the equation for $w(y,t)$ established in question 1:

\[
	w(y,t) = \frac{A}{\rho~i~n} \bigg [ 1 - \frac{ J_0(\alpha y  i^{\frac{3}{2}}) } { J_0(\alpha i^{\frac{3}{2}}) } \bigg ] e^{int}
\]
Substituting into the previous equation $n$ with $\alpha = R \sqrt{\frac{n}{\nu}}, n = \nu \frac{\alpha}{R}^2$

\[
	w(y,t) =  \frac{A~R^2}{i \rho \nu} \bigg [  \frac{  J_0(\alpha i^{\frac{3}{2}})  - J_0(\alpha y  i^{\frac{3}{2}}) } { \alpha^2 J_0(\alpha i^{\frac{3}{2}}) } \bigg ] e^{i \frac{\nu~t}{R^2} \alpha^2}	\\
\]

Let $B=\alpha y  i^{\frac{3}{2}}, C = \alpha i^{\frac{3}{2}}$ and $D=i \frac{\nu~t}{R^2} \alpha^2$, rewrite the previous equation
\[
	w(y,t) =  \frac{A~R^2}{i \rho \nu} \bigg [  \frac{  J_0(C)  - J_0(B) } { \alpha^2 J_0(C) } \bigg ] e^D	\\
\]
When $n \rightarrow 0, \alpha \rightarrow 0$ and we have the indeterminate form for $w(y,t) =  \frac{A~R^2}{i \rho \nu} ( \frac{ 1 - 1 } {0 \cdot 1} ) \cdot 1 = \frac{0}{0}$.
Therefore we apply L'Hospital's rule, compute the derivatives of numerator and denominator and taking the limit $\alpha \rightarrow 0$:
\ba
	\frac{d}{d \alpha} (J_0(C) - J_0(B)) e^D	&=	\frac{d}{d \alpha} (J_0(C) - J_0(B)) ~ e^D + (J_0(C) - J_0(B)) \frac{d}{d \alpha} e^D \\
	\frac{d}{d \alpha} (J_0(C) - J_0(B))		&= -i^{\frac{3}{2}} J_1(C) + i^{\frac{3}{2}} y J_1(B) \\
									&=  i^{\frac{3}{2}}  (y J_1(B) - J_1(C)) e^D \\
	\frac{d}{d \alpha} e^D				&= \frac{2 i \nu t}{R^2} \alpha e^D \\					
\ea
So
\ba
	\frac{d}{d \alpha} (J_0(C) - J_0(B)) e^D	&= \bigg ( i^{\frac{3}{2}}  (y J_1(B) - J_1(C)) +  (J_0(C) - J_0(B))  \frac{2 i \nu t}{R^2} \alpha \bigg ) e^D \\
	\frac{d} {d \alpha} \alpha^2 J_0(C)		&= 2 \alpha J_0(C) +  \alpha^2 i^{\frac{3}{2}} (- ~ J_1(C)) \\
									&= \alpha (2 J_0(C) - i^{\frac{3}{2}} \alpha J_1(C)) \\
\ea
And 
\ba
	\lim_{\alpha \rightarrow 0} \frac{d}{d \alpha} (J_0(C) - J_0(B)) e^D	&=  \bigg ( i^{\frac{3}{2}}  (y \cdot 0 - 0 ) + (0 - 0) \frac{2 i \nu t}{R^2} \cdot 0  \bigg ) 1 = 0 \\
	\lim_{\alpha \rightarrow 0} \frac{d} {d \alpha} \alpha^2 J_0(C)		&= 0 \cdot (2 \cdot 1 - i^{\frac{3}{2}} \cdot 0 \cdot 0) = 0 \\
\ea
We still have the indeterminate form $\frac{0}{0}$, so we apply one more time L'Hospital's rule
\ba
	\frac{d^2}{d \alpha^2} (J_0(C) - J_0(B)) e^D	&= \bigg (  i^{\frac{3}{2}} (y  \frac{d}{d \alpha}  J_1(B) - \frac{d}{d \alpha} J_1(C)) + ( \frac{d}{d \alpha} J_0(C) 
										- \frac{d}{d \alpha} J_0(B))  \frac{2 i \nu t}{R^2} \alpha \\
										& + (J_0(C) - J_0(B))  \frac{2 i \nu t}{R^2} \bigg ) e^D + \\
										& \bigg ( i^{\frac{3}{2}}  (y J_1(B) - J_1(C)) +  (J_0(C) - J_0(B))  \frac{2 i \nu t}{R^2} \alpha \bigg )  \frac{2 i \nu t}{R^2} \alpha e^D \\
										&=  \bigg (  i^{\frac{3}{2}} ( i^{\frac{3}{2}} y^2  \frac{J_0(B) - J_2(B)}{2} - i^{\frac{3}{2}} \frac{J_0(C) - J_2(C)}{2}) + \\
										& ( i^{\frac{3}{2}}  y J_1(B) -  i^{\frac{3}{2}} J_1(C))  \frac{2 i \nu t}{R^2} \alpha + (J_0(C) - J_0(B))  \frac{2 i \nu t}{R^2} \bigg ) e^D  + \\
										& \bigg ( i^{\frac{3}{2}}  (y J_1(B) - J_1(C)) +  (J_0(C) - J_0(B))  \frac{2 i \nu t}{R^2} \alpha \bigg )  \frac{2 i \nu t}{R^2} \alpha e^D \\
										&= \bigg ( \frac{i^3  y^2 } {2} (J_0(B) - J_2(B)) - \frac{ i^3}{2} (J_0(C) - J_2(C)) + \\
										& ( i^{\frac{3}{2}}  y J_1(B) -  i^{\frac{3}{2}} J_1(C))  \frac{2 i \nu t}{R^2} \alpha + (J_0(C) - J_0(B))  \frac{2 i \nu t}{R^2} \bigg ) e^D  + \\
										& \bigg ( i^{\frac{3}{2}}  (y J_1(B) - J_1(C)) +  (J_0(C) - J_0(B))  \frac{2 i \nu t}{R^2} \alpha \bigg )  \frac{2 i \nu t}{R^2} \alpha e^D \\
	\frac{d^2}{d \alpha^2} \alpha^2 J_0(C)		&= 2 J_0(C) - i^{\frac{3}{2}} \alpha J_1(C)  + \alpha (2 \frac{d}{d \alpha} J_0(C) \\
										& -  i^{\frac{3}{2}} J_1(C)  - i^{\frac{3}{2}} \alpha \frac{d}{d \alpha} J_1(C))  \\	
										&= 2 J_0(C) - i^{\frac{3}{2}} \alpha J_1(C)  + \alpha (2 i^{\frac{3}{2}} (-J_1(C)) \\
										& -  i^{\frac{3}{2}} J_1(C)  - i^3 \alpha \frac{J_0(C) - J_2(C)}{2} ) \\ 	
										&= 2 J_0(C) - i^{\frac{3}{2}} J_1(C)  - 3 i^{\frac{3}{2}} J_1(C) \alpha  - \frac{i^3}{2} (J_0(C) - J_2(C)) \alpha \\							
\ea
Next we take the second derivative of the numerator of $w(y,t)$ when $\alpha \rightarrow 0$ using the expression above:
\ba
	\lim_{\alpha \rightarrow 0} \frac{d^2}{d \alpha^2} (J_0(C) - J_0(B)) e^D	&= \bigg ( \frac{i^3 y^2 } {2} (1 -  0) - \frac{ i^3 }{2} (1 - 0) + \\
															& ( i^{\frac{3}{2}}  y \cdot 0 -  i^{\frac{3}{2}}  \cdot 0) \frac{2 i \nu t}{R^2} \cdot 0 + (1 - 1)  \frac{2 i \nu t}{R^2} \bigg )\cdot 1 + \\
															& \bigg ( i^{\frac{3}{2}}  (y \cdot 0 - 0) + (1- 1) \frac{2 i \nu t}{R^2}  \cdot 0 \bigg )  \frac{2 i \nu t}{R^2}  \cdot 0 \cdot 1 \\
															&= \frac{i^3} {2} (y^2 - 1) = \frac{i} {2} (1 - y^2) \\
\ea
Similarly for the limit of $\alpha \rightarrow 0$ of the second derivative of the denominator of  $w(y,t)$, we get
\ba
	\lim_{\alpha \rightarrow 0} \frac{d^2}{d \alpha^2}\alpha^2 J_0(C)		&= 2 \cdot 1 - i^{\frac{3}{2}} \cdot 0 - 3 i^{\frac{3}{2}}  \cdot 0 \cdot 0  - \frac{i^3}{2}  (1 - 0) \cdot 0\\
															&= 2 \\
\ea
Therefore for a constant input pressure, we have
\[
	w = \frac{A~R^2}{i \rho \nu} \frac{i}{4}  (1 - y^2) = \frac{A}{4 \mu} R^2 (1 - y^2)
\]
which is equation 2 in Wormersley’s paper with $A=\frac{p_1 - p_2}{l}$\\

Using the expression of the differential equation established in question (1) and with $n=0$, we want to solve
\[
	r^2 \frac{d^2 u(r)}{d r^2} + r \frac{d u(r)}{d r}  = - \frac{A}{\mu} r^2
\]
This is an Euler equation or a Legendre ordinary differential equation  with $\alpha=1, \beta=0$, so we make the change of variable $e^t= r$ or $\ln{r}= t$.
Then $r \frac{du}{dr} = \frac{d u}{d t}$ and $r^2 \frac{d^2y}{dr^2} = \frac{d^2u}{dt^2} - \frac{du}{dt}$.

which yields for the ODE
\ba
	\frac{d^2 u}{d t^2} - \frac{du}{dt} +  \frac{du}{dt}	&= - \frac{A}{\mu} e^{2t} \\
	\frac{d^2 u}{d t^2} &= - \frac{A}{\mu} e^{2t} \\
\ea
For the homogeneous equation integrating twice gives $u(t) = C_1 t + C_2$ or $u(r) = C_1 \ln(r) + C_2$.
Take for one particular solution of the ODE:
$u_p(t) = C_3 e^{2t}, u'_p(t) = 2 C_3 e^{2t}, u''_p(t) = 4 C_3 e^{2t}$, 
substitute into the ODE gives $4 C_3 e^{2t} =  - \frac{A}{\mu} e^{2t}$ or $C_3 = - \frac{A}{4 \mu}$ 
thus $u_p(t) =  - \frac{A}{4 \mu} e^{2t}$ or $u_p(r) =  - \frac{A}{4 \mu} r^2$.
Therefore the total solution is 
\[
	u(r) = - \frac{A}{4 \mu} r^2  + C_1 \ln(r) + C_2
\]
From this, we write $u'(r) = - \frac{A}{2 \mu} r  + \frac{C_1}{r}$.
So to have the boundary condition $\frac{\partial w} {\partial r} |_{r=0} = 0$ or $\frac{\partial u} {\partial r} |_{r=0} = 0$, $C_1$ has to be zero so $u'(0)$ is finite and equal to $0$.
The second boundary condition $w(R)=0$, or $u(R)=0$, gives $C_2 = \frac{A}{4 \mu} R^2$.
Finally
\[
	u(r) = -  \frac{A}{4 \mu}(r^2 - R^2) = \frac{A}{4 \mu} R^2 (1 - (\frac{r}{R})^2) =  \frac{A}{4 \mu} R^2 (1 - y^2)
\]
which is equation (2) in Wormersley’s paper with $n=0, A=\frac{p_1 - p_2}{l}$ and which is the expression of $w$ we just found in question 3.


\section*{Question 4}
For Poiseuille's flow
\[
	w = \frac{p_1 - p_2}{4 \mu l} R^2 (1 - y^2)
\]
And
\[
	Q = 2 \pi \int_0^R w(r,t) r dr
\]


Make the change of variable $y = \frac{r}{R}, dy = \frac{dr}{R}$ and we have
\ba
	Q	&= 2 \pi \int_0^1  \frac{p_1 - p_2}{4 \mu l} R^2 (1 - y^2) R ~ y ~ R ~ dy \\
		&= 2 \pi  \frac{p_1 - p_2}{4 \mu l} R^4  \int_0^1  (1 - y^2) ~ y ~ dy \\
		&= 2 \pi  \frac{p_1 - p_2}{4 \mu l} R^4  [\frac{y^2}{2} - \frac{y^4}{4} ]_0^1 \\
		&= 2 \pi  \frac{p_1 - p_2}{4 \mu l} R^4  \frac{1}{4} \\
		&=  \frac{p_1 - p_2}{8 \mu l} \pi R^4 
\ea
$=$ equation 20 in Wormersley’s paper.


\end{document}
